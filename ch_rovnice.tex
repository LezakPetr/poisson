\chapter{Rovnice}

Poissonova parciální diferenciální rovnice má velké využití v technické praxi, protože pomocí ní lze modelovat mnoho důležitých jevů.
Rovnici lze zapsat ve více různých ekvivalentních tvarech zapsaných v~\eqref{eq:poissonova_rovnice_definice}. V~kartézské soustavě souřadnic na přímce má rovnice tvar \eqref{eq:poissonova_rovnice_primka}, v~rovině
\eqref{eq:poissonova_rovnice_rovina} a~v~prostoru \eqref{eq:poissonova_rovnice_prostor}. Tyto tvary rovnic nazýváme diferenciální.

\begin{equation}
\label{eq:poissonova_rovnice_definice}
\begin{split}
\Delta \varphi = f \\
\diverg \ \grad \ \varphi = f
\end{split}
\end{equation}

\begin{equation}
\label{eq:poissonova_rovnice_primka}
\frac{\partial^2 \varphi}{\partial x^2} = f
\end{equation}

\begin{equation}
\label{eq:poissonova_rovnice_rovina}
\frac{\partial^2 \varphi}{\partial x^2} + \frac{\partial^2 \varphi}{\partial y^2} = f
\end{equation}

\begin{equation}
\label{eq:poissonova_rovnice_prostor}
\frac{\partial^2 \varphi}{\partial x^2} + \frac{\partial^2 \varphi}{\partial y^2} + \frac{\partial^2 \varphi}{\partial z^2} = f
\end{equation}

Rovnici by samozřejmě bylo možné zobecnit i~do vícerozměrných prostorů, my se však v dalším výkladu zaměříme pouze na nejvýše tři rozměry.
Zde \(\varphi\) je skalární potenciál. Skalární pole \(f\) je pole zdrojů, někdy též nazývané buzení.

Poissonovu rovnici lze také zapsat v~integrálním tvaru. Rovnici \eqref{eq:poissonova_rovnice_integralni_tvar} lze z rovnice \eqref{eq:poissonova_rovnice_definice} odvodit Gaussovým teorémem.

\begin{equation}
\label{eq:poissonova_rovnice_integralni_tvar}
\begin{split}
\oint_{\partial V} \left( \grad \ \varphi \right) \cdot \vect{n} ds = \int_V f dV
\end{split}
\end{equation}

Speciálním případem Poissonovy rovnice bez zdrojů je rovnice Laplaceova. V uvažované oblasti je \(f = 0\), rovnice má tvar \eqref{eq:laplaceova_rovnice_definice}. Obdobně integrální tvar rovnice má tvar \eqref{eq:laplaceova_rovnice_integralni_tvar}.

\begin{equation}
\label{eq:laplaceova_rovnice_definice}
\begin{split}
\Delta \varphi = f \\
\diverg \ \grad \ \varphi = 0
\end{split}
\end{equation}

\begin{equation}
\label{eq:laplaceova_rovnice_integralni_tvar}
\begin{split}
\oint_{\partial V} \left( \grad \ \varphi \right) \cdot \vect{n} ds = \int_V f dV
\end{split}
\end{equation}

\begin{fact}
Rovnicím bez zdrojů říkáme homogenní. Rovnicím se zdroji říkáme nehomogenní. Laplaceova rovnice je tedy homogenní rovnice, Poissonova
nehomogenní.
\end{fact}

