\chapter{Souřadné systémy}

V~této příloze jsou vyjádřeny operace v~bežně používaných souřadných systémech.

\section{Kartézský souřadný systém}

\section{Polární souřadný systém}

\begin{equation}
\begin{split}
x = r \cdot \cos \alpha \\
y = r \cdot \sin \alpha
\end{split}
\end{equation}

\begin{equation}
\begin{split}
r = \sqrt{x^2 + y^2} \\
\alpha = \arctg \frac{y}{x}
\end{split}
\end{equation}



\section{Cylindrický souřadný systém}

\begin{equation}
\begin{split}
x = r \cdot \cos \alpha \\
y = r \cdot \sin \alpha \\
z = z
\end{split}
\end{equation}

\begin{equation}
\begin{split}
r = \sqrt{x^2 + y^2} \\
\alpha = \arctg \frac{y}{x} \\
z = z
\end{split}
\end{equation}



\section{Sférický souřadný systém}

\begin{equation}
\begin{split}
x = r \cdot \cos \beta \cdot \cos \alpha \\
y = r \cdot \cos \beta \cdot \sin \alpha \\
z = r \cdot \sin \beta
\end{split}
\end{equation}

\begin{equation}
\begin{split}
r = \sqrt{x^2 + y^2 + z^2} \\
\alpha = \arctg \frac{y}{x} \\
\beta = \arctg \frac{z}{\sqrt{x^2 + y^2}}
\end{split}
\end{equation}
