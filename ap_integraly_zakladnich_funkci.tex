\chapter{Integrály základních funkcí}
\label{ap:integraly_zakladnich_funkci}

V~této příloze jsou sespsány integrály základních funkcí. Jedná se o~inverzní vztahy ke vztahům uvedeným v~příloze~\ref{ap:derivace_zakladnich_funkci}.

\begin{equation}
\int 0 \cdot \mathrm{d}x = \mathrm{C}
\end{equation}

\begin{equation}
\int 1 \cdot \mathrm{d}x = x + \mathrm{C}
\end{equation}

\begin{equation}
\int (\mathrm{f}(x) \pm \mathrm{g}(x)) \cdot \mathrm{d}x = \int \mathrm{f}(x) \cdot \mathrm{d}x \pm \int \mathrm{g}(x) \cdot \mathrm{d}x
\end{equation}

\begin{equation}
\int \mathrm{f}(x) \cdot \frac{\partial \mathrm{g}}{\partial x} \cdot \mathrm{d}x = \mathrm{f}(x) \cdot \mathrm{g}(x) - \int \mathrm{g}(x) \cdot \frac{\partial \mathrm{f}}{\partial x} \cdot \mathrm{d}x
\end{equation}

\begin{equation}
\int k \cdot \mathrm{f}(x) \cdot \mathrm{d}x = k \cdot \int \mathrm{f}(x) \cdot \mathrm{d}x
\end{equation}

\begin{equation}
\int x^n \cdot \mathrm{d}x = \frac{x^{n + 1}}{n + 1} + \mathrm{C}; n \in \mathbb{N}_0
\end{equation}

\begin{equation}
\int x^r \cdot \mathrm{d}x = \frac{x^{r + 1}}{r + 1} + \mathrm{C}; x > 0, r \neq -1
\end{equation}

\begin{equation}
\int e^x \cdot \mathrm{d}x = e^x + \mathrm{C}
\end{equation}

\begin{equation}
\int a^x \cdot \mathrm{d}x = \frac{a^x}{\ln a} + \mathrm{C}
\end{equation}

\begin{equation}
\int \frac{1}{x} \cdot \mathrm{d}x = \ln x  + \mathrm{C}; x > 0
\end{equation}
