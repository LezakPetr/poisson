\chapter{Integrály základních funkcí}
\label{ap:integraly_zakladnich_funkci}

\begin{prolog}
:- ensure_loaded("../equations/formula").

make_test_real_numbers([-2, -1, 0, 1, 1.5, 2, 3]).
make_test_natural_numbers([1, 2, 3, 4, 5]).
make_test_real_functions(FX, [FX, FX^2, FX + 1, (FX + 1) * FX^5]).
\end{prolog}

V~této příloze jsou sespsány integrály základních funkcí. Jedná se o~inverzní vztahy ke vztahům uvedeným v~příloze~\ref{ap:derivace_zakladnich_funkci}.

\begin{prolog}
?-	make_test_real_numbers(Numbers),
	print_validated_formula(
		"ap_integral_zero",
		declare(
			[
				variable(X, "x", Numbers),
				variable(C, "C", Numbers)
			],
			equal_transform([derivative(X)], [
				integral(X, 0), C
			])
		)
	).
\end{prolog}
\eeq{ap_integral_zero}

\begin{prolog}
?-	make_test_real_numbers(Numbers),
	print_validated_formula(
		"ap_integral_one",
		declare(
			[
				variable(X, "x", Numbers),
				variable(C, "C", Numbers)
			],
			equal_transform([derivative(X)], [
				integral(X, 1), X + C
			])
		)
	).
\end{prolog}
\eeq{ap_integral_one}

\begin{prolog}
?-	make_test_real_numbers(Numbers),
	make_test_real_functions(FX, FunctionsF),
	make_test_real_functions(GX, FunctionsG),
	print_validated_formula(
		"ap_integral_plus_minus",
		declare(
			[
				variable(X, "x", Numbers),
				function(F, "f", FunctionsF),
				function(G, "g", FunctionsG),
				plus_minus(PM)
			],
			equal_transform([derivative(X)], [
				integral(X, plus_minus(apply(F, [FX], [X]), apply(G, [GX], [X]), PM)),
				plus_minus(integral(X, apply(F, [FX], [X])), integral(X, apply(G, [GX], [X])), PM)
			])
		)
	).
\end{prolog}
\eeq{ap_integral_plus_minus}


\begin{prolog}
?-	make_test_real_numbers(Numbers),
	make_test_real_functions(FX, FunctionsF),
	make_test_real_functions(GX, FunctionsG),
	print_validated_formula(
		"ap_integral_multiply",
		declare(
			[
				variable(X, "x", Numbers),
				function(F, "f", FunctionsF),
				function(G, "g", FunctionsG)
			],
			equal_transform([derivative(X)], [
				integral(X, apply(F, [FX], [X]) * derivative(X, apply(G, [GX], [X]))),
				apply(F, [FX], [X]) * apply(G, [GX], [X]) - integral(X, derivative(X, apply(F, [FX], [X])) * apply(G, [GX], [X]))
			])
		)
	).
\end{prolog}
\eeq{ap_integral_multiply}

\begin{prolog}
?-	make_test_real_numbers(Numbers),
	make_test_real_functions(FX, FunctionsF),
	print_validated_formula(
		"ap_integral_multiply_by_constant",
		declare(
			[
				variable(X, "x", Numbers),
				variable(K, "k", Numbers),
				function(F, "f", FunctionsF)
			],
			equal_transform([derivative(X)], [
				integral(X, K * apply(F, [FX], [X])),
				K * integral(X, apply(F, [FX], [X]))
			])
		)
	).
\end{prolog}
\eeq{ap_integral_multiply_by_constant}

\begin{prolog}
?-	make_test_real_numbers(Numbers),
	print_validated_formula(
		"ap_integral_power_by_natural",
		declare(
			[
				variable(X, "x", Numbers),
				variable(C, "C", Numbers)
			],
			forall_in(N, "n", nonnegative_integers, Numbers,
				equal_transform([derivative(X)], [
					integral(X, X^N),
					X^(N + 1) / (N + 1) + C
				])
			)
		)
	).
\end{prolog}
\eeq{ap_integral_power_by_natural}

\begin{prolog}
?-	make_test_real_numbers(Numbers),
	print_validated_formula(
		"ap_integral_power_by_real",
		declare(
			[
				variable(C, "C", Numbers)
			],
			forall_in(X, "x", positive_real_numbers, Numbers,
				forall_in(R, "r", difference(real_numbers, set_of([-1])), Numbers,
					equal_transform([derivative(X)], [
						integral(X, X^R),
						X^(R + 1) / (R + 1) + C
					])
				)
			)
		)
	).
\end{prolog}
\eeq{ap_integral_power_by_real}

\begin{prolog}
?-	make_test_real_numbers(Numbers),
	print_validated_formula(
		"ap_integral_exp",
		declare(
			[
				variable(X, "x", Numbers),
				variable(C, "C", Numbers)
			],
			equal_transform([derivative(X)], [
				integral(X, e^X),
				e^X + C
			])
		)
	).
\end{prolog}
\eeq{ap_integral_exp}

\begin{prolog}
?-	make_test_real_numbers(Numbers),
	print_validated_formula(
		"ap_integral_general_power",
		declare(
			[
				variable(X, "x", Numbers),
				variable(C, "C", Numbers)
			],
			forall_in(A, "a", difference(positive_real_numbers, set_of([1])), Numbers,
				equal_transform([derivative(X)], [
					integral(X, A^X),
					A^X / ln(A) + C
				])
			)
		)
	).
\end{prolog}
\eeq{ap_integral_general_power}


\begin{prolog}
?-	make_test_real_numbers(Numbers),
	print_validated_formula(
		"ap_integral_inverse",
		declare(
			[
				variable(C, "C", Numbers)
			],
			forall_in(X, "x", positive_real_numbers, Numbers,
				equal_transform([derivative(X)], [
					integral(X, 1 / X),
					ln(X) + C
				])
			)
		)
	).
\end{prolog}
\eeq{ap_integral_inverse}
