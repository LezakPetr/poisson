\documentclass{book}
\usepackage[utf8]{inputenc}
\usepackage[czech]{babel}
\usepackage{amsmath}
\usepackage{tikz}
\usetikzlibrary{math}

\newcommand{\vect}[1]{\boldsymbol{#1}}
\newcommand{\grad}{\mathrm{grad}}
\newcommand{\diverg}{\mathrm{div}}
\newcommand{\rot}{\mathrm{rot}}

\title{Poissonova parciální diferenciální rovnice [draft]}
\author{Petr Ležák}

\begin{document}
\maketitle

\chapter{Copyright}

\input{README.txt}

\chapter{Úvod}

\chapter{Vektorová analýza}

\section{Základní definice}

Mějme třírozměrný euklidovský prostor. Tělesem \(\Psi\) rozumíme souvislou ohraničenou množinu bodů v tomto prostoru. Hranicí tělesa rozumíme množinu bodů, které sousedí jak s~body tvořícími těleso tak i~s~body netvořícími těleso. Hranici označujeme \(\partial \Psi\). Obdobně definujeme útvar v~rovině nebo úsek na přímce. Těleso, rovinný útvar nebo úsek obecně nazveme oblast. Hranicí tělesa je uzavřená plocha, hranicí plochy je uzavřená křivka a~hranicí úseku jsou 2 body. 

\begin{tikzpicture}
\filldraw[lightgray] (0, 0) circle (1);
\draw (0, 0) circle (1);
\draw (0, 0) node[anchor=center]{\(\Psi\)};
\draw (1, 0) node[anchor=west]{\(\partial \Psi\)};
\end{tikzpicture}

Skalární funkcí (polem) \(u(x, y)\) respektive \(u(x, y, z)\) rozumíme funkci přiřazující každému bodu \([x, y]\) v~rovině respektive \([x, y, z]\) v~prostoru skalární hodnotu. Příkladem takovéto funkce je například pole teploty nebo tlaku. Pokud není uvedeno jinak, tak uvažujeme kartézský systém souřadnic. 

Vektorovou funkcí (polem) \(\vect{u}(x, y)\) respektive \(\vect{u}(x, y, z)\) rozumíme funkci přiřazující každému bodu \([x, y]\)
 v~rovině respektive \([x, y, z]\) v~prostoru vektorovou hodnotu. Příkladem takovéto funkce je například pole rychlosti proudění kapaliny. Funkci můžeme zapsat ve složkovém tvaru \(\vect{u}=(u_x(x, y, z), u_y(x, y, z), u_z(x, y, z))\).

\section{Křivkový integrál}

Pokud máme orientovanou po částech hladkou křivku v~prostoru definovanou parametricky pomocí rovnice \eqref{eq:krivkovy_integral_krivka} a vektorové pole \(\vect{v}\), pak křivkový integrál definujeme rovnicí \eqref{eq:krivkovy_integral_definice}.

\begin{equation}
\label{eq:krivkovy_integral_krivka}
\Gamma = (\Gamma_x(t), \Gamma_y(t), \Gamma_z(t)), t \in <t_1, t_2>
\end{equation}

\begin{equation}
\label{eq:krivkovy_integral_definice}
\begin{split}
\int_\Gamma \vect{v} \cdot d\vect{l}
\end{split}
\end{equation}

Protože \(\frac{d\vect{l}}{dt} = \left(\frac{d \Gamma_x}{dt}, \frac{d \Gamma_y}{dt}, \frac{d \Gamma_z}{dt} \right)\), tak pro parametrickou křivku
\(\Gamma\) platí vztah \eqref{eq:krivkovy_integral_vypocet}.

\begin{equation}
\label{eq:krivkovy_integral_vypocet}
\begin{split}
\int_\Gamma \vect{v} \cdot d\vect{l} =
\int_{t_1}^{t_2} \left(v_x(\Gamma(t)) \cdot \frac{d \Gamma_x}{dt} + v_y(\Gamma(t)) \cdot \frac{d \Gamma_y}{dt} + v_z(\Gamma(t)) \cdot \frac{d \Gamma_z}{dt} \right) dt
\end{split}
\end{equation}

Obdobně je definován křivkový integrál v rovině, pouze vynecháme souřadnici \(z\).

Je-li křivka uzavřená (její počáteční a koncový bod jsou shodné), toto zdůrazníme kroužkem přes symbol integrálu, tedy \(\oint_\Gamma \vect{v} \cdot d\vect{l}\), a říkáme mu cirkulace vektoru po uzavřené smyčce.

Z definice \eqref{eq:krivkovy_integral_definice} plyne, že integrujeme skalární součin hodnoty pole \(v\) a elementu dráhy \(dl\),
tedy součin průmětu pole do dráhy a elementu dráhy. Křivkový integrál proto lze použít pro výpočet práce, energie a podobně.


\subsection{Příklad}

Elektrické pole má intenzitu \(E = \left(-\frac{y}{x^2+y^2}, \frac{x}{x^2+y^2}\right) \frac{\mathrm{N}}{\mathrm{C}}\). Jakou práci pole vykoná na jednotkovém náboji, který se pohybuje po dráze \(\Gamma(t) = (1, t), t \in <-1, 1>\)?


\begin{tikzpicture}
\draw[->] (-2.5, 0) -- (2.5, 0) node[anchor=south east]{x};
\draw[->] (0, -2.5) -- (0, 2.5) node[anchor=north west]{y};

\foreach \r in {0.3679, 0.6065, 1, 1.6487}
	\draw[->, thin] (\r, 0) arc (0:360:\r);
	
\draw (1.65, 0) node[anchor=south west]{\(\vect{E}\)};
	
\draw[->, thick] (1, -1) -- (1, 1) node[anchor=north west]{\(\Gamma\)};
\end{tikzpicture}


Elektrické pole působí na náboj \(Q\) silou \(\vect{F} = \vect{E} \cdot Q\). Práce vykonaná přesunutím náboje \(Q\) po dráze \(\vect{ds}\) je tedy \(dW = \vect{F} \cdot \vect{ds} = \vect{E} \cdot Q \cdot \vect{ds}\) a pro jednotkový náboj \(dW = \vect{E} \cdot \vect{ds}\). Celková práce je tedy

\[
W = \int_{-1}^1 \left (\frac{-t}{1 + t^2} \cdot 0 + \frac{1}{1 + t^2} \cdot 1 \right) = \left[\mathrm{arctg} \ t\right]_{-1}^1 = \frac{\pi}{2} \mathrm{J}
\]

Nyní vyšetřeme vlastnosti křivkového integrálu. Rovnice \eqref{eq:krivkovy_integral_opacny} říká, že křivkový integrál po opačně orientované křivce je opačný. Rovnice \eqref{eq:krivkovy_integral_soucet} udává, že křivkový integrál po křivce, která je rovna součtu dvou křivek, je rovna součtu jejich křivkových integrálů. Oba dva vztahy jsou intuitivní a~lze je snadno ověřit rozepsáním integrálů pomocí vztahu \eqref{eq:krivkovy_integral_vypocet}.

\begin{equation}
\label{eq:krivkovy_integral_opacny}
\begin{split}
\int_{-\Gamma} \vect{v} \cdot d\vect{l} = -\int_{\Gamma} \vect{v} \cdot d\vect{l}
\end{split}
\end{equation}

\begin{equation}
\label{eq:krivkovy_integral_soucet}
\begin{split}
\int_{\Gamma_1 + \Gamma_2} \vect{v} \cdot d\vect{l} = \int_{\Gamma_1} \vect{v} \cdot d\vect{l} + \int_{\Gamma_2} \vect{v} \cdot d\vect{l}
\end{split}
\end{equation}

Máme-li uzavřenou smyčku \(\Gamma_S\), pak ji můžeme rozdělit na dvě smyčky \(\Gamma_1\) a~\(\Gamma_2\) pomocí libovolně volené přepážky \(\Gamma_P\).

\begin{tikzpicture}
\tikzmath{ \x = 0; \y = 5; }

\draw[->] (\x + 2, \y) arc (0:360:2 and 1);
\draw (\x + 2, \y) node[anchor=west]{\(\Gamma_S\)};

\tikzmath{ \x = 0; \y = 2.5; \t = 0.1; }

\draw[->] (\x + \t, \y - 1) arc (-90:90:2 and 1);
\draw[->] (\x + \t, \y + 1) -- (\x + \t, \y - 0.9);

\draw[->] (\x - \t, \y + 1) arc (90:270:2 and 1);
\draw[->] (\x - \t, \y - 0.9) -- (\x - \t, \y + 1);
\draw (\x, \y) node[anchor=west]{\(\Gamma_P\)};

\draw (\x + \t, \y - 0.9) -- (\x - \t, \y - 0.9);
\draw (\x - \t, \y - 1) -- (\x + \t, \y - 1);

\tikzmath{ \x = 0; \y = 0; \t = 0.1; }

\draw[->] (\x + \t, \y - 1) arc (-90:90:2 and 1);
\draw[->] (\x + \t, \y + 1) -- (\x + \t, \y - 1);
\draw (\x - 2, \y + 0) node[anchor=east]{\(\Gamma_1\)};

\draw[->] (\x - \t, \y + 1) arc (90:270:2 and 1);
\draw[->] (\x - \t, \y - 1) -- (\x - \t, \y + 1);
\draw (\x + 2, \y) node[anchor=west]{\(\Gamma_2\)};
\end{tikzpicture}

Přepážka je tvořena dvěma opačně orientovanými částmi křivky, které jsou nekonečně blízké. Proto se jejich příspěvky do cirkulace vyruší a platí proto vztah \eqref{eq:deleni_cirkulace}. Takto lze cirkulaci vektoru podél křivky rozdělit na součet libovolného počtu cirkulací.

\begin{equation}
\label{eq:deleni_cirkulace}
\begin{split}
\oint_{\Gamma_S} \vect{v} \cdot d\vect{l} = \oint_{\Gamma_1} \vect{v} \cdot d\vect{l} + \oint_{\Gamma_2} \vect{v} \cdot d\vect{l}
\end{split}
\end{equation}

\subsection{Rotace}

Vyšetřeme nyní cirkulaci vektoru nekonečně malou smyčkou. Vektorové pole \(\vect{v}\) linearizujeme podle vztahu \eqref{eq:rotace_linearizace}.

\begin{equation}
\label{eq:rotace_linearizace}
\vect{v} \approx \begin{pmatrix}
v_{x_0} + \frac{\partial v_x}{\partial x} \cdot x + \frac{\partial v_x}{\partial y} \cdot y + \frac{\partial v_x}{\partial z} \cdot z \\
v_{y_0} + \frac{\partial v_y}{\partial x} \cdot x + \frac{\partial v_y}{\partial y} \cdot y + \frac{\partial v_y}{\partial z} \cdot z \\
v_{z_0} + \frac{\partial z_y}{\partial x} \cdot x + \frac{\partial v_z}{\partial y} \cdot y + \frac{\partial v_z}{\partial z} \cdot z
\end{pmatrix}
\end{equation}

Dosadíme-li linearizaci \eqref{eq:rotace_linearizace} do vztahu \eqref{eq:krivkovy_integral_vypocet}, získáme vztah \eqref{eq:rotace_1}.

\begin{equation}
\label{eq:rotace_1}
\int_\Gamma \vect{v} \cdot d\vect{l} \approx
\int_{t_1}^{t_2} \begin{pmatrix}
\left(v_{x_0} + \frac{\partial v_x}{\partial x} \cdot \Gamma_x(t) + \frac{\partial v_x}{\partial y} \cdot \Gamma_y(t) + \frac{\partial v_x}{\partial z} \cdot \Gamma_z(t)\right) \cdot \frac{d \Gamma_x}{dt} + \\
\left(v_{y_0} + \frac{\partial v_y}{\partial x} \cdot \Gamma_x(t) + \frac{\partial v_y}{\partial y} \cdot \Gamma_y(t) + \frac{\partial v_y}{\partial z} \cdot \Gamma_z(t)\right) \cdot \frac{d \Gamma_y}{dt} + \\
\left(v_{z_0} + \frac{\partial z_y}{\partial x} \cdot \Gamma_x(t) + \frac{\partial v_z}{\partial y} \cdot \Gamma_y(t) + \frac{\partial v_z}{\partial z} \cdot \Gamma_z(t)\right) \cdot \frac{d \Gamma_z}{dt}
\end{pmatrix} dt
\end{equation}

Úpravou získáme vztah \eqref{eq:rotace_2}.

\begin{equation}
\label{eq:rotace_2}
\begin{matrix}
v_{x_0} \int_{t_1}^{t_2} \frac{d \Gamma_x}{dt} dt + \frac{\partial v_x}{\partial x} \int_{t_1}^{t_2} \Gamma_x(t) \frac{d \Gamma_x}{dt} dt + \frac{\partial v_x}{\partial y} \int_{t_1}^{t_2} \Gamma_y(t) \frac{d \Gamma_x}{dt} dt + \frac{\partial v_x}{\partial z} \int_{t_1}^{t_2} \Gamma_z(t) \frac{d \Gamma_x}{dt} dt + \\
v_{y_0} \int_{t_1}^{t_2} \frac{d \Gamma_y}{dt} dt + \frac{\partial v_y}{\partial x} \int_{t_1}^{t_2} \Gamma_x(t) \frac{d \Gamma_y}{dt} dt + \frac{\partial v_y}{\partial y} \int_{t_1}^{t_2} \Gamma_y(t) \frac{d \Gamma_y}{dt} dt + \frac{\partial v_y}{\partial z} \int_{t_1}^{t_2} \Gamma_z(t) \frac{d \Gamma_y}{dt} dt + \\
v_{z_0} \int_{t_1}^{t_2} \frac{d \Gamma_z}{dt} dt + \frac{\partial z_y}{\partial x} \int_{t_1}^{t_2} \Gamma_x(t) \frac{d \Gamma_z}{dt} dt + \frac{\partial v_z}{\partial y} \int_{t_1}^{t_2} \Gamma_y(t) \frac{d \Gamma_z}{dt} dt + \frac{\partial v_z}{\partial z} \int_{t_1}^{t_2} \Gamma_z(t) \frac{d \Gamma_z}{dt} dt
\end{matrix}
\end{equation}

Vyšetřeme nyní první řádek součtu. Ostatní řádky jsou obdobné.

Integrály \(\int_{t_1}^{t_2} \frac{d \Gamma_x}{dt} dt\) a \(\int_{t_1}^{t_2} \Gamma_x(t) \frac{d \Gamma_x}{dt} dt\) jsou integrály typu \(\int_{t_1}^{t_2} \mathrm{f}(\Gamma_x(t)) \frac{d \Gamma_x}{dt} dt\). Zavedeme-li substituci \(x = \Gamma_x\), tak se integrál redukuje na \(\int_{\Gamma_x(t_1)}^{\Gamma_x(t_2)} \mathrm{f}(x) dx\) a protože je smyčka \(\Gamma\) uzavřená, tak jsou obě integrační meze shodné, tedy \(\Gamma_x(t_1) = \Gamma_x(t_2)\). Proto pro uzavřenou smyčku \(\Gamma\) platí \(\int_{t_1}^{t_2} \mathrm{f}(\Gamma_x(t)) \frac{d \Gamma_x}{dt} dt = 0\).

Mějme nekonečně malou plochu \(S\) v~prostoru, na ní ležící bod \(P\) a~vektorové pole \(\vect{v}\). 

Rotace vektorového pole \(\vect{v}\) je vektorové pole pole \(v\). Je definována vztahem \(\rot \ \vect{u} = \rot \ (u_x(x, y, z), u_y(x, y, z), u_z(x, y, z)) = \left(\frac{\partial u_z}{\partial y} - \frac{\partial u_y}{\partial z}, \frac{\partial u_x}{\partial z} - \frac{\partial u_z}{\partial x}, \frac{\partial u_y}{\partial x} - \frac{\partial u_x}{\partial y}\right)\).


\section{Tok vektoru}

Pokud máme vektorové pole definované v~prostoru, tak můžeme určit jeho tok určenou plochou. Obdobně pokud máme vektorové pole definované
 v~rovině, tak můžeme určit jeho tok určenou křivkou. V~obou případech je tento tok roven integrálu průmětů vektorů pole do normál plochy nebo křivky násobené plochou nebo délkou elementu plochy nebo křivky.
 
Pokud vektorové pole představuje rychlost proudění tekutiny, pak tok vektoru představuje celkový průtok této tekutiny plochou nebo křivkou.

\subsection{Tok vektoru plochou v prostoru - plošný integrál}

Je-li \(S = \left(S_x(t, u), S_y(t, u), S_z(t, u)\right)\) po částech hladká plocha s jednotkovou normálou \(\vect{n}\) a \(\vect{v}\) je vektorové pole,
pak plošný integrál definujeme rovnicí \eqref{eq:plosny_integral_definice}. Říkáme mu tok vektoru \(v\) plochou \(S\).


\begin{equation}
\label{eq:plosny_integral_definice}
\begin{split}
\int_S \vect{v} \cdot d\vect{s} = \int_S \vect{v} \cdot \vect{n} ds
\end{split}
\end{equation}


Povšimněme si, že vektory \(\frac{\partial S}{\partial t}\) a \(\frac{\partial S}{\partial u}\) jsou (v daném bodě) tečné k ploše \(S\). Proto vektorový součin \(\frac{\partial S}{\partial t} \times \frac{\partial S}{\partial u}\) má směr normály. Ne však nutně shodnou orientaci. Navíc \(\frac{\partial S}{\partial t} \times \frac{\partial S}{\partial u} dt \ du\) má velikot splošného elementu \(ds\). Proto lze plošný integrál až na znaménko spočítat pomocí vztahu \eqref{eq:plosny_integral_vypocet}.

\begin{equation}
\label{eq:plosny_integral_vypocet}
\begin{split}
\int_S \vect{v} \cdot \vect{n} ds = \iint_S \vect{v} \cdot \left (\frac{\partial S}{\partial t} \times \frac{\partial S}{\partial u}\right) dt \ du
\end{split}
\end{equation}

Je-li plocha uzavřená, pak toto zdůrazníme kroužkem přes symbol integrálu, tedy \(\oint_S \vect{v} \cdot d\vect{s}\). Podle konvence pak normála \(n\) míří vně z plochy.

Například zákon kontinuity toku nestlačitelné kapaliny lze zapsat \(\oint_S \vect{v} \cdot d\vect{s} = 0\). Vyjadřuje fakt, že velkový výtok vektoru rychlosti kapaliny \(v\) jakoukoli uzavřenou plochou \(S\) je nulový, tedy že se kapalina nikde neztrácí, nevyvěrá ani nestlačuje. 

\subsection{Příklad}

Rychlost kapaliny je popsána vztahem \(v = \)

\subsection{Tok vektoru křivkou v~rovině}


Je-li \(l = \left(l_x(t), l_y(t)\right)\) po částech hladká plocha s jednotkovou normálou \(\vect{n}\) a \(\vect{v}\) je vektorové pole,
pak tok vektoru \(v\) křivkou \(l\) definujeme rovnicí \eqref{eq:tok_krivkou_definice}.

\begin{equation}
\label{eq:tok_krivkou_definice}
\begin{split}
\int_l \vect{v} \cdot \vect{n} dl
\end{split}
\end{equation}

Máme-li vektor \(d \vect{l} = \left(\frac{d l_x}{dt}, \frac{d l_y}{dt}\right)\), pak vektor k~němu kolmý získáme tak, že prohodíme souřadnice a~u~jedné z~nich otočíme znaménko. Proto \(\vect{n} dl = \left(\frac{d l_y}{dt}, -\frac{d l_x}{dt}\right) dt\). Tok vektoru \(v\) křivkou \(l\) proto můžeme až na znaménko vypočítat podle vztahu \eqref{eq:tok_krivkou_vypocet}.

\begin{equation}
\label{eq:tok_krivkou_vypocet}
\begin{split}
\int_{t_1}^{t_2} \left (v_x \cdot \frac{d l_y}{dt} -v_y \cdot \frac{d l_x}{dt} \right) dt
\end{split}
\end{equation}

\section{Operátory}

\subsection{Gradient}

Gradient skalárního pole \(\varphi\) je vektorové pole \(\vect{v}\), které má v~každém bodě směr nejvyššího přírůstku skalárního pole. Je definován vztahem
\[
\vect{v} = \grad \ \varphi = \left(\frac{\partial \varphi}{x}, \frac{\partial \varphi}{y}, \frac{\partial \varphi}{z}\right)
\].

\subsection{Divergence}

Divergence vektorového pole \(\vect{v}\) je skalární pole \(k\), které má v~každém bodě hodnotu celkového výtoku z~infinitezimálního objemu okolo daného bodu. V prostoru je definována vztahem \(\diverg \ \vect{u} = \diverg (u_x(x, y, z), u_y(x, y, z), u_z(x, y, z)) = \frac{\partial u_x}{\partial x} + \frac{\partial u_y}{\partial y} + \frac{\partial u_z}{\partial z}\).
V rovině je definována vztahem \(\diverg \ \vect{u} = \diverg (u_x(x, y), u_y(x, y)) = \frac{\partial u_x}{\partial x} + \frac{\partial u_y}{\partial y}\).
 
\begin{tikzpicture}
\draw (-1, -1) -- (-1, 1) -- (1, 1) -- (1, -1) -- (-1, -1);
\draw[->] (1, 0) -- (2, 0);
\draw[->] (-1, 0) -- (-2, 0);
\draw[->] (0, 1) -- (0, 2);
\draw[->] (0, -1) -- (0, -2);
\filldraw (0, 0) circle (1pt) node[anchor=south]{[x, y]};
\draw (0, -1) node[anchor=south]{dx};
\draw (1, 0) node[anchor=east]{dy};
\draw (1, 0) node[anchor=south west]{\(d_y \cdot u_x(x + \frac{dx}{2}, y)\)};
\draw (-1, 0) node[anchor=south east]{\(-d_y \cdot u_x(x - \frac{dx}{2}, y)\)};
\draw (0, 1) node[anchor=south west]{\(d_x \cdot u_y(x, y + \frac{dy}{2})\)};
\draw (0, -1) node[anchor=north west]{\(-d_x \cdot u_y(x, y - \frac{dy}{2})\)};
\end{tikzpicture}

\subsection{Laplaceův operátor}

Laplaceův operátor \(\Delta\) přiřazuje skalárnímu poli \(\varphi\) skalární pole \(k\) podle vztahu \(\Delta \varphi = \diverg \ \grad \ \varphi = \left(\frac{\partial^2 \varphi}{\partial x^2}, \frac{\partial^2 \varphi}{\partial y^2}, \frac{\partial^2 \varphi}{\partial z^2}\right)\). Operátor může působit i na vektorové pole, v tom případě působí na každou složku zvlášť.

\subsection{Vztahy mezi operátory}

Gradient, divergence, rotace i aplikace laplaceova operátoru jsou lineární operace. Nechť \(\alpha\) a \(\beta\) jsou konstanty, \(\varphi\) a \(rho\) skalární pole a \(u\) a \(v\) vektorová pole. Pak platí tyto vztahy:
\[
\grad(\alpha \cdot \varphi + \beta \cdot \rho) = \alpha \cdot \grad \ \varphi + \beta \cdot \grad \ \rho
\]
\[
\diverg(\alpha \cdot \vect{u} + \beta \cdot \vect{v}) = \alpha \cdot \diverg \ \vect{u} + \beta \cdot \diverg \ \vect{v}
\]
\[
\rot(\alpha \cdot \vect{u} + \beta \cdot \vect{v}) = \alpha \cdot \rot \ \vect{u} + \beta \cdot \rot \ \vect{v}
\]

Tyto vztahy lze jednoduše ověřit dosazením do jejich definic.


\section{Gaussův teorém}

\begin{equation}
\int_V \diverg \ u \ dV = \oint_{\partial V} \vect{u} \cdot \vect{n} ds
\end{equation}


Greenův teorém:

\begin{equation}
\int_S \diverg \ u \ dV = \oint_{\partial S} \vect{u} \cdot \vect{n} dl
\end{equation}

\chapter{Rovnice}

Poissonova parciální diferenciální rovnice má velké využití v technické praxi, protože pomocí ní lze modelovat mnoho důležitých jevů.
Rovnici lze zapsat ve více různých ekvivalentních tvarech zapsaných v~\eqref{eq:poissonova_rovnice_definice}. V~kartézské soustavě souřadnic na přímce má rovnice tvar \eqref{eq:poissonova_rovnice_primka}, v~rovině
\eqref{eq:poissonova_rovnice_rovina} a~v~prostoru \eqref{eq:poissonova_rovnice_prostor}. Tyto tvary rovnic nazýváme diferenciální.

\begin{equation}
\label{eq:poissonova_rovnice_definice}
\begin{split}
\Delta \varphi = f \\
\nabla^2 \varphi = f \\
\diverg \ \grad \ \varphi = f
\end{split}
\end{equation}

\begin{equation}
\label{eq:poissonova_rovnice_primka}
\frac{\partial^2 \varphi}{\partial x^2} = f
\end{equation}

\begin{equation}
\label{eq:poissonova_rovnice_rovina}
\frac{\partial^2 \varphi}{\partial x^2} + \frac{\partial^2 \varphi}{\partial y^2} = f
\end{equation}

\begin{equation}
\label{eq:poissonova_rovnice_prostor}
\frac{\partial^2 \varphi}{\partial x^2} + \frac{\partial^2 \varphi}{\partial y^2} + \frac{\partial^2 \varphi}{\partial z^2} = f
\end{equation}

Rovnici by samozřejmě bylo možné zobecnit i~do vícerozměrných prostorů, my se však v dalším výkladu zaměříme pouze na nejvýše tři rozměry.
Zde \(varphi\) je skalární potenciál. Jedná se o~skalární pole, tedy o~funkci definovanou v~určité oblasti (prostoru, rovině, přímce případně v~jejich částech), která každému bodu této oblasti přiřazuje skalární
hodnotu. Skalární pole \(f\) je pole zdrojů, někdy též nazývané buzení.

Poissonovu rovnici lze také zapsat v~integrálním tvaru.

\begin{equation}
\label{eq:poissonova_rovnice_integralni_tvar}
\begin{split}
\oint_{\partial V} \grad \ \varphi \cdot \vect{n} ds = \int_V f dV
\end{split}
\end{equation}


elektrické pole stacionárních nábojů
magnetické pole neměnných elektrických proudů
ustálený tok
tok mechanického napětí v materiálu
difuze
proud v odporovém materiálu
Základní vlastnosti


\chapter{Řešení Poissonovy rovnice bez okrajových podmínek}

Pokud řešíme Poissonovu rovnici na nekonečné rovině nebo v nekonečném prostoru, pak zde nejsou zadané žádné okrajové podmínky. Začneme odvozením
potenciálu bodového zdroje.

\section{Potenciál kruhového disku v rovině}

Mějme kruhový disk o~poloměru \(R\) s konstantním buzením \(f\). Chceme vypočítat potenciál \(\varphi\). Konfigurace je rotačně symetrická vůči středu
disku, proto i řešení rovnice je rotačně symetrické. Potenciál proto závisí pouze na vzdálenosti od středu disku \(r\).


\begin{tikzpicture}
\draw (0, 0) circle(1);
\draw[->] (0, 0) -- (1, 0);
\draw (0.5, 0) node[anchor=south]{R};

\draw[dashed] (0, 0) circle(2);
\draw[->] (0, 0) -- (0, 2);
\draw (0, 1) node[anchor=south west]{r};

\end{tikzpicture}


\begin{equation}
\begin{split}
\diverg \ \grad \varphi = f
\end{split}
\end{equation}

Na rovnici aplikujeme Greenův teorém.

\begin{equation}
\begin{split}
\int_S f \ ds = \int_S \diverg \ \grad \varphi \ ds = \int_{\partial S} (\grad \varphi) \cdot \vect{n} \ dl
\end{split}
\end{equation}

Protože potenciál závisí pouze na \(r\), tak složka gradientu kolmá na poloměr je nulová. Proto \((\grad \varphi) \cdot \vect{n} = \frac{d \varphi}{dr}\).
Získáme tak rovnici \eqref{eq:potencial_disku_obecne}.

\begin{equation}
\label{eq:potencial_disku_obecne}
\begin{split}
\int_S f \ ds = \int_{\partial S} \frac{d \varphi}{dr} \ dl = 2 \pi r \frac{d \varphi}{dr}
\end{split}
\end{equation}

Označme celkové buzení \(F = \pi R^2 f\).
Potenciál uvnitř a vně disku vyřešíme odděleně.

\subsection{Potenciál vně disku}

Tato situace nastává pokud \(r \geq R\). Uvnitř plochy \(S\) se proto nachází celý disk. Rovnici \eqref{eq:potencial_disku_obecne} proto můžeme updavit na tvar \eqref{eq:potencial_disku_vne_0}.

\begin{equation}
\label{eq:potencial_disku_vne_0}
\begin{split}
F = 2 \pi r \frac{d \varphi}{dr}
\end{split}
\end{equation}

To je obyčejná diferenciální rovnice se separovatelnými proměnnými:

\begin{equation}
\label{eq:potencial_disku_vne_1}
\begin{split}
d \varphi = \frac{F}{2 \pi r} dr \\
\int d \varphi = \int \frac{F}{2 \pi r} dr \\
\varphi = \frac{F}{2 \pi} \cdot \mathrm{ln} \ r + C_1 \\
\end{split}
\end{equation}

Konstantu \(C_1\) jsme zvolíme 0 a získáme tak rovnici \eqref{eq:potencial_disku_vne_2}. Tuto rovnici můžeme využít i pro bodový zdroj, kdy \(R \rightarrow 0\).

\begin{equation}
\label{eq:potencial_disku_vne_2}
\varphi = \frac{F}{2 \pi} \cdot \mathrm{ln} \ r
\end{equation}

\subsection{Potenciál uvnitř disku}

Tato situace nastává pokud \(r \leq R\). Uvnitř plochy \(S\) se proto nachází pouze část disku - celá plocha je vyplněna diskem.

Vyjádříme \(f\) podle \(F\).

\begin{equation}
f = \frac{F}{\pi R^2}
\end{equation}

Integrál na levé straně rovnice \eqref{eq:potencial_disku_obecne} nahradíme vztahem pro obsah kruhu a dosadíme za \(f\).

\begin{equation}
\begin{split}
\pi r^2 f = 2 \pi r \frac{d \varphi}{dr} \\
\pi r^2 \cdot \frac{F}{\pi R^2} = 2 \pi r \frac{d \varphi}{dr} \\
F \cdot \frac{r^2}{R^2} = 2 \pi r \frac{d \varphi}{dr}
\end{split}
\end{equation}

To je obyčejná diferenciální rovnice se separovatelnými proměnnými:

\begin{equation}
\label{eq:potencial_disku_uvnitr}
\begin{split}
d \varphi = F \cdot \frac{r}{2 \pi R^2} dr \\
\int d \varphi = \int F \cdot \frac{r}{2 \pi R^2} dr \\
\varphi = F \cdot \frac{r^2}{4 \pi R^2} + C_2
\end{split}
\end{equation}

Konstantu \(C_2\) určíme tak, aby potenciál na okraji disku (\(r = R\)) navazoval na potenciál vně disku:

\begin{equation}
\begin{split}
F \cdot \frac{R^2}{4 \pi R^2} + C_2 = \frac{F}{2 \pi} \cdot \mathrm{ln} \ R \\
C_2 = \frac{F}{2 \pi} \cdot \mathrm{ln} \ R - \frac{F}{4 \pi}
\end{split}
\end{equation}

Potenciál uvnitř disku je tedy

\begin{equation}
\begin{split}
\varphi = F \cdot \frac{r^2}{4 \pi R^2} + \frac{F}{2 \pi} \cdot \mathrm{ln} \ R - \frac{F}{4 \pi} \\
\varphi = \frac{F}{4 \pi} \cdot \left (\frac{r^2}{R^2} + 2 \cdot \mathrm{ln} \ R - 1 \right)
\end{split}
\end{equation}


\section{Potenciál koule v prostoru}

Mějme kouli o~poloměru \(R\) s konstantním buzením \(f\). Chceme vypočítat potenciál \(\varphi\). Konfigurace je rotačně symetrická vůči středu
koule, proto i řešení rovnice je rotačně symetrické. Potenciál proto závisí pouze na vzdálenosti od středu koule \(r\).


\begin{tikzpicture}
\draw (0, 0) circle(1);
\draw[->] (0, 0) -- (1, 0);
\draw (0.5, 0) node[anchor=south]{R};

\draw[dashed] (0, 0) circle(2);
\draw[->] (0, 0) -- (0, 2);
\draw (0, 1) node[anchor=south west]{r};

\end{tikzpicture}


\begin{equation}
\begin{split}
\diverg \ \grad \varphi = f
\end{split}
\end{equation}

Na rovnici aplikujeme Gaussův teorém.

\begin{equation}
\begin{split}
\int_V f \ ds = \int_V \diverg \ \grad \varphi \ ds = \int_{\partial V} (\grad \varphi) \cdot \vect{n} \ ds
\end{split}
\end{equation}

Protože potenciál závisí pouze na \(r\), tak složka gradientu kolmá na poloměr je nulová. Proto \((\grad \varphi) \cdot \vect{n} = \frac{d \varphi}{dr}\).
Získáme tak rovnici \eqref{eq:potencial_koule_obecne}.

\begin{equation}
\label{eq:potencial_koule_obecne}
\begin{split}
\int_V f \ ds = \int_{\partial V} \frac{d \varphi}{dr} \ dl = 4 \pi r^2 \frac{d \varphi}{dr}
\end{split}
\end{equation}

Označme celkové buzení \(F = \frac{4}{3} \pi R^3 f\).
Potenciál uvnitř a vně koule vyřešíme odděleně.

\subsection{Potenciál vně koule}

Tato situace nastává pokud \(r \geq R\). Uvnitř objemu \(V\) se proto nachází celá koule. Rovnici \eqref{eq:potencial_koule_obecne} proto můžeme updavit na tvar \eqref{eq:potencial_koule_vne_0}.

\begin{equation}
\label{eq:potencial_koule_vne_0}
\begin{split}
F = 4 \pi r^2 \frac{d \varphi}{dr}
\end{split}
\end{equation}

To je obyčejná diferenciální rovnice se separovatelnými proměnnými:

\begin{equation}
\label{eq:potencial_koule_vne_1}
\begin{split}
d \varphi = \frac{F}{4 \pi r^2} dr \\
\int d \varphi = \int \frac{F}{4 \pi r^2} dr \\
\varphi = -\frac{F}{4 \pi r} + C_1 \\
\end{split}
\end{equation}

Konstantu \(C_1\) jsme zvolíme 0 a získáme tak rovnici \eqref{eq:potencial_kould_vne_2}. Touto volbou zajistíme, že potenciál v nekonečnu je nulový. Konstantu lze ale zvolit jakkoli. Rovnici \eqref{eq:potencial_kould_vne_2} můžeme využít i pro bodový zdroj, kdy \(R \rightarrow 0\).

\begin{equation}
\label{eq:potencial_koule_vne_2}
\varphi = -\frac{F}{4 \pi r}
\end{equation}

\subsection{Potenciál uvnitř koule}

Tato situace nastává pokud \(r \leq R\). Uvnitř objemu \(V\) se proto nachází pouze část koule - celý objem je vyplněn koulí.

Vyjádříme \(f\) podle \(F\).

\begin{equation}
f = \frac{3}{4 \pi R^3}
\end{equation}

Integrál na levé straně rovnice \eqref{eq:potencial_koule_obecne} nahradíme vztahem pro objem koule a dosadíme za \(f\).

\begin{equation}
\begin{split}
\frac{4}{3} \pi r^3 f = 4 \pi r^2 \frac{d \varphi}{dr} \\
\frac{4}{3} \pi r^3 \frac{3}{4 \pi R^3} F = 4 \pi r^2 \frac{d \varphi}{dr} \\
\frac{r^3}{R^3} F = 4 \pi r^2 \frac{d \varphi}{dr} \\
\frac{r}{R^3} F = 4 \pi \frac{d \varphi}{dr}
\end{split}
\end{equation}

To je obyčejná diferenciální rovnice se separovatelnými proměnnými:

\begin{equation}
\label{eq:potencial_disku_uvnitr}
\begin{split}
d \varphi = \frac{r}{4 \pi R^3} F \ dr \\
\varphi = \frac{r^2}{8 \pi R^3} F + C_2
\end{split}
\end{equation}

Konstantu \(C_2\) určíme tak, aby potenciál na okraji koule (\(r = R\)) navazoval na potenciál vně koule:

\begin{equation}
\begin{split}
\frac{R^2}{8 \pi R^3} F + C_2 = -\frac{F}{4 \pi R} \\
C_2 = -\frac{F}{4 \pi R} - \frac{R^2}{8 \pi R^3} F \\
C_2 = -F \cdot \frac{3}{8 \pi R}
\end{split}
\end{equation}

Potenciál uvnitř koule je tedy

\begin{equation}
\begin{split}
\varphi = \frac{r^2}{8 \pi R^3} F - F \cdot \frac{3}{8 \pi R} \\
\varphi = \frac{F}{8 \pi R} \cdot \left(\frac{r^2}{R^2} - 3 \right) \\
\end{split}
\end{equation}


\section{Potenciál v nekonečné rovině bez okrajových podmínek}

Rovnici \eqref{eq:potencial_disku_vne_2} lze využít pro výpočet potenciálu obecného rozložení zdrojů v nekonečné rovině.

Mějme rozdělení hustoty zdrojů \(f(\vect{A})\). Integrováním rovnice \eqref{eq:potencial_disku_vne_2} pak získáme rovnici \eqref{eq:potencial_v_nekonecne_rovine}.

\begin{equation}
\label{eq:potencial_v_nekonecne_rovine}
\varphi(\vect{B}) = \frac{1}{2 \pi} \cdot \int f(\vect{A}) \mathrm{ln} |\vect{A} - \vect{B}| \ d\vect{A}
\end{equation}

\section{Potenciál v nekonečném prostoru bez okrajových podmínek}

Rovnici \eqref{eq:potencial_koule_vne_2} lze využít pro výpočet potenciálu obecného rozložení zdrojů v nekonečném prostoru.

Mějme rozdělení hustoty zdrojů \(f(\vect{A})\). Integrováním rovnice \eqref{eq:potencial_koule_vne_2} pak získáme rovnici \eqref{eq:potencial_v_nekonecnem_prostoru}.

\begin{equation}
\label{eq:potencial_v_nekonecnem_prostoru}
\varphi(\vect{B}) = -\frac{1}{4 \pi} \int \frac{f(\vect{A})}{|\vect{A} - \vect{B}|} \ d\vect{A}
\end{equation}

\subsection{Příklad - vzájemná indukčnost válcových cívek}

Mějme 2 souosé hustě vinuté válcové cívky \(L_1\) a \(L_2\) mající \(N_1\) a \(N_2\) závitů. Cívky jsou na ose mezi souřadnicemi \(x_{11}\) až \(x_{12}\) a \(x_{21}\) až \(x_{22}\) a na poloměrech \(R_{11}\) až \(R_{12}\) a \(R_{21}\) až \(R_{22}\). Cívkou \(L_1\) protéká časově proměnný proud \(i_1(t)\) který v prostoru vyvolá magnetické pole. Toto magnetické pole indukuje v cívce \(L_2\) časově proměnné napětí \(u_2(t)\). Vzájemná indukčnost je definována vztahem \eqref{eq:definice_m}.

\begin{equation}
\label{eq:definice_m}
u_2 = M \cdot \frac{di_1}{dt}
\end{equation}

Pokud bychom chtěli spočítat vlastní indukčnost \(L\), pak stačí zvolit obě cívky shodné, tedy \(L_1 = L_2\).


\begin{tikzpicture}
\draw[dashdotted] (0, 0) -- (10, 0);
\draw (1, 0.5) -- (1, -0.5);
\draw (4, 0.5) -- (4, -0.5);
\filldraw[color=black, fill=gray] (1, 0.5) -- (1, 1) -- (4, 1) -- (4, 0.5) -- (1, 0.5);
\filldraw[color=black, fill=gray] (1, -0.5) -- (1, -1) -- (4, -1) -- (4, -0.5) -- (1, -0.5);

\draw (6, 1) -- (6, -1);
\draw (8, 1) -- (8, -1);
\filldraw[color=black, fill=gray] (6, 1) -- (6, 2) -- (8, 2) -- (8, 1) -- (6, 1);
\filldraw[color=black, fill=gray] (6, -1) -- (6, -2) -- (8, -2) -- (8, -1) -- (6, -1);

\end{tikzpicture}

Podle indukčního zákona platí vztah \eqref{eq:indukcni_zakon}.

\begin{equation}
\label{eq:indukcni_zakon}
u_2 = \frac{d\Phi_2}{dt}
\end{equation}

Sloučením vztahů \label{eq:definice_m} a \eqref{eq:indukcni_zakon} získáme vztah \eqref{eq:vypocet_m_obecne}.

\begin{equation}
\label{eq:vypocet_m_obecne}
\begin{split}
M \cdot \frac{di_1}{dt} = \frac{d\Phi_2}{dt} \\
M = \frac{\Phi_2}{i_1} = \frac{\int_{S_2} \vect{B} \cdot {n} ds}{i_1}
\end{split}
\end{equation}

Plocha \(S_2\) je libovolná plocha ohraničená vodičem cívky \(L_2\), pro všechny plochy ohraničené stejným vodičem vyjde integrál stejný.

Cívku \(L_2\) si zjednodušíme jako \(N_2\) kruhových závitů ležících v rovinách kolmých na osu cívky. Protože je cívka hustě vinutá, tak je závitů mnoho a jsou po cívce rovnoměrně rozděleny. Proto si můžeme dovolit namísto počítání jednotlivých závitů zvlášť považovat závity za kontinuum. Závity proložíme kruhy, které jsou kolmé na osu cívky. Proto nás bude zajímat pouze složka indukčnosti rovnoběžná s osou cívky. Tok magnetické indukčnosti kruhy ohraničenými závity cívky nahradíme cirkulací magnetického potenciálu závity. Získáme tak vztah \eqref{eq:aproximace_plochy}. Zde \(\varphi\) je složka magnetického potenciálu rovnoběžná s kruhovým závitem. Díky rotační symetrii úlohy předpokládáme, že tato složka je po celé délce závitu stejná.

Označme plochy poloviny průřezů \(S_1 = (R_{12} - R_{11}) \cdot (L_{12} - L_{11})\) a \(S_2 = (R_{22} - R_{21}) \cdot (L_{22} - L_{21})\).

\begin{equation}
\label{eq:aproximace_plochy}
\begin{split}
M = \int_{S_2} \vect{B} \cdot {n} \ ds = \int_{\partial S_2} \varphi \ dl_2 \approx \\
\frac{N_2}{S_2} \int_{R_{21}}^{R_{22}} \int_{L_{21}}^{L_{22}} \int_{0}^{2 \pi} r_2 \varphi \ d\alpha \ dl_2 \ dr_2 = \\
\frac{N_2}{S_2} \int_{R_{21}}^{R_{22}} \int_{L_{21}}^{L_{22}} 2 \pi r_2 \cdot \varphi(r_2, l_2) \ dl_2 \ dr_2 
\end{split}
\end{equation}

Dále musíme vyjádřit magnetický potenciál \(\varphi\):

\begin{equation}
\label{eq:civky_potencial}
\begin{split}
\varphi(r_2, l_2) = \frac{N_1 \mu}{S_1} \int_{0}^{2 \pi} \int_{R_{11}}^{R_{12}} \int_{L_{11}}^{L_{12}} \frac{r_1}{4 \pi r}  \ dl_1 \ dr_1 \ d \alpha
\end{split}
\end{equation}

\(r\) je vzdálenost elementu cívky \(L_1\) od elementu cívky \(L_2\). Vektor je \(\vect{r} = (l_2 - l_1, r_2 - r_1 \cdot cos \alpha, r1 \cdot sin \alpha)\), proto \(r = \sqrt{(l_2 - l_1)^2 + (r_2 - r_1 \cdot cos \alpha)^2 + r_1^2 \cdot sin^2 \alpha}\).

Indukčnost je tedy

\begin{equation}
\label{eq:indukcnost_1}
\begin{split}
M = \frac{N_1 N_2 \mu}{S_1 S_2} \int_{R_{21}}^{R_{22}} \int_{L_{21}}^{L_{22}} \int_{0}^{2 \pi} \int_{R_{11}}^{R_{12}} \int_{L_{11}}^{L_{12}} \frac{r_1 r_2}{2 r} \ dl_1 \ dr_1 \ d \alpha \ dl_2 \ dr_2
\end{split}
\end{equation}

Poissonova rovnice

Lagrangeova rovnice

Linearita

Invariance vůči posunutí a pootočení

Okrajové podmínky

Řešení rovnice na přímce

Řešení Lagrangeovy rovnice uvnitř obecné smyčky

Přílohy

\begin{thebibliography}{9}

\bibitem{pum1}
Karel Rektoris a spolupracovníci,
\textit{Přehled užité matematiky I},
Prometheus,
ISBN 80-7196-180-9

\bibitem{eam}
Bedřich Sedlák, Ivan Štoll,
\textit{Elektřina a magnetismus},
Academia,
ISBN 80-200-1004-1

\end{thebibliography}

\end{document}

