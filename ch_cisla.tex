\chapter{Čísla}

\begin{prolog}
:- ensure_loaded("../equations/formula").
:- ensure_loaded("../equations/truth_table").

make_test_numbers([-1, 0, 1, 2, 3, 1.5]).
make_test_predicates(Y, [num_equal([Y, 1], 0), num_equal([Y, 2], 0), log_true, log_false]).
\end{prolog}


\begin{abstract}
V~této kapitole definujeme obory čísel a~prozkoumáme jejich vlastnosti.
\end{abstract}

\section{Přirozená čísla}

Začněme nějjednoduššími tzv. přirozenými čísly. Přirozenými čísly rozumíme čísla 1, 2, 3 atd. Upozorňuji, že literatura není jednotná v~tom, zda je nula přirozené číslo. V~této knize nulu nepovažujeme za přirozené číslo. Přirozená čísla typicky vyjadřují počet nějakých objektů. Množinu všech přirozených čísel značíme \(\natural\). Tato množina je nekonečná, neexistuje největší přirozené číslo.

Prozkoumejme, jak můžeme přirozená čísla definovat. Nejnižší přirozené číslo je 1. Každé přirozené číslo \(n\) má
následovníka, označme ho \(n + 1\). Naopak můžeme říci, že přirozené číslo je buď 1, nebo je následník jiného 
přirozeného čísla. To nás vede k definici:

\begin{prolog}
?-	make_test_numbers(Values),
	print_validated_formula(
		'natural_numbers_definition',
		declare_variable(N, 'n', Values,
			equiv(
				in(N, natural_numbers),
				or(
					num_equal([N, 1], 0),
					exists_in(M, 'm', natural_numbers, Values, num_equal([M + 1, N], 0)) 
				)
			)
		)
	).
\end{prolog}
\eeq{natural_numbers_definition}.

Číslo 2 tedy můžeme zapsat jako \(1 + 1\), číslo 3 jako \(2 + 1 = (1 + 1) + 1\) atd. Zápis a~význam přirozených čísel nám shrnuje tabulka~\ref{tab:natural_numbers}. Vidíme, že počet jedniček v~zápisu odpovídá počtu reprezentovaných objektů.

\begin{table}[ht]
\centering
\begin{tabular}{|c|c|c|}
Číslo & Zápis & Význam \\
1 & 1 & \(\bigcirc\) \\
2 & 1 + 1 & \(\bigcirc \bigcirc\) \\
3 & (1 + 1) + 1 & \(\bigcirc \bigcirc \bigcirc\) \\
4 & ((1 + 1) + 1) + 1 & \(\bigcirc \bigcirc \bigcirc \bigcirc\)
\end{tabular}
\caption{Přirozená čísla}
\label{tab:natural_numbers}
\end{table}


Fakt, že každé přirozené číslo kromě jedničky je následník jiného přirozeného čísla, nám umožňuje zavést důkaz matematickou indukcí. Máme-li predikát \(\predicate{A}(n)\), který je pravdivý pro \(n=1\) a~z~jeho pravdivosti pro \(n\) plyne  pravdivost pro \(n + 1\), pak je tento predikát pravdivý pro všechna přirozená čísla. Důkaz matematickou
indukcí popisuje vztah~\eqref{eq:mathematical_induction}.

\begin{prolog}
?-	make_test_numbers(Values),
	make_test_predicates(Z, Predicates),
	print_validated_formula(
		'mathematical_induction',
		declare_predicate(A, 'A', Predicates,
			impl(			
				and(
					apply(A, [Z], [1]),
					forall_in(N, 'n', natural_numbers, Values,
						impl(
							apply(A, [Z], [N]),
							apply(A, [Z], [N + 1])
						)
					)
				),
				forall_in(N, 'n', natural_numbers, Values,
					apply(A, [Z], [N])
				)
			)
		)
	).
\end{prolog}
\eeq{mathematical_induction}.

Intuitivně je princip důkazu matematickou indukcí zřejmý. Víme, že predikát \(\predicate{A}(n)\) je pravdivý pro \(n=1\). Z~jeho pravdivosti pro \(n=1\) plyne pravdivost pro \(n=2\), z~ní pak pro \(n=3\) atd. Jedná se vlastně o~tranzitivitu implikace prodlouženou do nekonečna. Obdobně lze dokazovat platnost predikátů pro přirozená čísla, jen nezačínáme nulou ale jedničkou.

Nyní, když už víme, co jsou přirozená čísla, tak si zkusme definovat jejich součet. 