\chapter{Komplexní čísla}


\begin{prolog}
:- ensure_loaded("../equations/formula").
:- ensure_loaded("../equations/truth_table").
:- ensure_loaded("../equations/draw_1d_function").

make_test_numbers([-1, 0, 0.5, 1, 2, 3, 1.5]).
make_test_nonzero_numbers([-1, 1, 2, 3, 1.5]).
make_test_positive_numbers([1, 2, 3, 1.5]).
make_test_log_bases([0.5, 2, 3, 1.5]).
make_test_integers([-3, -2, -1, 0, 1, 2, 3]).
make_test_nonzero_integers([-3, -2, -1, 1, 2, 3]).
make_test_natural_numbers([1, 2, 3, 4, 5]).
make_test_predicates(Y, [equal([Y, 1]), equal([Y, 2]), log_true, log_false]).

make_test_nonempty_sets([set(S1), set(S2), set(S3), set(S4)]) :-
	make_set([1], S1),
	make_set([2], S2),
	make_set([1, 2], S3),
	make_set([-1, 0, 1, 2, 3, 1.5], S4).
	
\end{prolog}

V~sekci \ref{sec:realna_cisla} věnované reálným číslům jsme větou~\eqref{eq:sqrt_def} definovali odmocninu kladných čísel. Důvodem tohoto omezení je fakt, že v oboru reálných čísel neexistují sudé odmocniny ze záporných čísel. Například rovnice \(x^2 = -1\) nemá v~oboru reálných čísel řešení, neexistuje proto \(\sqrt{-1}\). Abychom mohli takovéto rovnice řešit, tak rovnicí~\eqref{eq:imag_def} zavedeme imaginární jednotku \(\imag\).

\begin{prolog}
?-	print_validated_formula(
		'imag_def',
		equal(imag^2, -1)
	).
\end{prolog}
\eeq{imag_def}
