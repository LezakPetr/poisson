\chapter{Komplexní čísla}


\begin{prolog}
:- ensure_loaded("../equations/formula").
:- ensure_loaded("../equations/truth_table").
:- ensure_loaded("../equations/draw_1d_function").

make_test_numbers([-1, 0, 0.5, 1, 2, 3, 1.5, complex(0, 1), complex(0, -1), complex(1, 1)]).
make_test_real_pairs([[0, 1]]).%, [0, -1], [1, 1], [0, 0], [3, 5], [-7, 4]]).
make_test_nonzero_real_pairs([[0, 1], [0, -1], [1, 1], [3, 5], [-7, 4]]).
make_test_complex_functions(Z, [Z + 2, Z - imag, (1 - imag) * Z, Z^2]).
make_test_real_binary_functions(X, Y, [X + Y]).%[X, Y, X + Y, X * Y, X^2 * Y, Y^2 * X]).
\end{prolog}

\section{Zavedení komplexních čísel}

V~sekci \ref{sec:realna_cisla} věnované reálným číslům jsme větou~\eqref{eq:sqrt_def} definovali odmocninu kladných čísel. Důvodem tohoto omezení je fakt, že v oboru reálných čísel neexistují sudé odmocniny ze záporných čísel. Například rovnice \(x^2 = -1\) nemá v~oboru reálných čísel řešení, neexistuje proto \(\sqrt{-1}\). Abychom mohli takovéto rovnice řešit, tak rovnicí~\eqref{eq:imag_def} zavedeme imaginární jednotku \(\imag\).

\begin{fact}
\begin{prolog}
?-	print_validated_formula(
		'imag_def',
		equal(imag^2, -1)
	).
\end{prolog}
\eeq{imag_def}
\end{fact}

Uvažujme dvě reálná čísla \(a\) a~\(b\), z~nichž alespoň jedno je nenulové. Chtějme nalézt takovou kombinaci těchto čísel, aby platilo:

\begin{prolog}
?-	make_test_numbers(Numbers),
	print_validated_formula(
		'complex_units_ortogonality',
		declare([variable(A, 'a', Numbers), variable(B, 'b', Numbers)],
			proof([
				equal(A * 1 + B * imag, 0)
			],
			[
				equal(A * 1, -B * imag),
				equal(A^2 * 1^2, B^2 * imag^2),
				equal(A^2, -B^2),
				equal(A^2 + B^2, 0)
			])
		)
	).
\end{prolog}
\eeq{complex_units_ortogonality}

Vidíme, že pro nenulová čísla tato rovnice nemůže být splněna. Reálná jednotka \(1\) a~imaginární jednotka \(\imag\) jsou proto navzájem ortogonální - jednu není možné vyjádřit pomocí druhé. Tyto jednotky si tedy můžeme představit jako vzájemně kolmé jednotkové osy ve dvojrozměrném prostoru. Každý bod v~tomto prostoru můžeme popsat jako lineární kombinaci těchto jednotek. Proto tuto lineární kombinaci \(a + b \cdot \imag\) nazveme komplexním číslem zapsaným v~algebraickém tvaru. Množinu všech komplexních čísel označíme \(\complex\).

\begin{fact}
\begin{prolog}
?-	make_test_numbers(Numbers),
	print_validated_formula(
		'complex_def',
		forall_in([A, B], ['a', 'b'], real_numbers, Numbers,
			in(A + B * imag, complex_numbers)
		)
	).
\end{prolog}
\eeq{complex_def}
\end{fact}

Vyjádřeme operace s~komplexními čísly. Využijeme přitom faktu, že se složkami můžeme provádět algebraické operace. Začněme součtem a~rozdílem komplexních čísel. Ten provádíme po složkách:

\begin{prolog}
?-	make_test_numbers(Numbers),
	print_validated_formula(
		'complex_add',
		declare([
			plus_minus(PM), variable(A, 'a', Numbers), variable(B, 'b', Numbers),
			variable(C, 'c', Numbers), variable(D, 'd', Numbers)
		],
			equal(
				plus_minus(par(A + B * imag), par(C + D * imag), PM),
				par(plus_minus(A, C, PM)) + par(plus_minus(B, D, PM)) * imag
			)
		)
	).
\end{prolog}
\eeq{complex_add}

Součin dvou komplexních čísel můžeme vyjádřit pomocí roznásobení výrazu:

\begin{prolog}
?-	make_test_numbers(Numbers),
	print_validated_formula(
		'complex_times',
		declare([variable(A, 'a', Numbers), variable(B, 'b', Numbers), variable(C, 'c', Numbers), variable(D, 'd', Numbers)],
			equal([
				(A + B * imag) * (C + D * imag),
				A * C + A * D * imag + B * C * imag + B * D * imag^2,
				linebreak,
				par(A * C - B * D) + (A * D  + B * C) * imag
			])
		)
	).
\end{prolog}
\eeq{complex_times}

Ke každému komplexnímu číslu \(a + b \cdot \imag\) můžeme zavést komplexně sdružené číslo \(a - b \cdot \imag\). Povšimněme si, že jejich součin daný vztahem~\eqref{eq:complex_multiply_by_conjugate} je reálný.

\begin{prolog}
?-	make_test_real_pairs(Numbers),
	print_validated_formula(
		'complex_multiply_by_conjugate',
		declare([variable([A, B], ['a', 'b'], Numbers)],
			equal(
				(A + B * imag) * (A - B * imag),
				A^2 + B^2
			)
		)
	).
\end{prolog}
\eeq{complex_multiply_by_conjugate}

Toho můžeme využít pro vyjádření podílu dvou komplexních čísel:

\begin{prolog}
?-	make_test_real_pairs(Pairs),
	make_test_nonzero_real_pairs(NonzeroPairs),
	print_validated_formula(
		'complex_fraction',
		declare([variable([A, B], ['a', 'b'], Pairs), variable([C, D], ['c', 'd'], NonzeroPairs)],
			equal([
				(A + B * imag) / (C + D * imag),
				((A + B * imag) * (C - D * imag)) / ((C + D * imag) * (C - D * imag)),
				linebreak,
				(A * C + B * D + B * C * imag - A * D * imag) / (C^2 + D^2),
				(A * C + B * D) / (C^2 + D^2) + (B * C - A * D) / (C^2 + D^2) * imag
			])
		)
	).
\end{prolog}
\eeq{complex_fraction}

Absolutní hodnota komplexního čísla \(x + Y \cdot \imag\) je dána výrazem \eqref{eq:complex_abs}. Jeho srovnáním se vztahem \eqref{eq:euklidovsky_prostor_metrika} dospějeme k~závěru, že komplexní rovina odpovídá dvojrozměrnému Euklidovskému prostoru.

\begin{prolog}
?-	make_test_real_pairs(Pairs),
	print_validated_formula(
		'complex_abs',
		declare([variable([X, Y], ['x', 'y'], Pairs)],
			equal(abs(X + Y * imag), sqrt(X^2 + Y^2))
		)
	).
\end{prolog}
\eeq{complex_abs}

\section{Komplexní funkce, limita, derivace}

Mějme komplexní funkci komplexní proměnné \(z\):

\begin{equation}
w = \func{f}(z); z, w \in \complex
\end{equation}

Tuto funkci můžeme rozepsat na dvě reálné funkce dvou reálných proměnných:

\begin{equation}
\func{f}(x + y \cdot \imag) = \func{u}(x, y) + \func{v}(x, y) \cdot \imag
\end{equation}

Limitu komplexní funkce definujeme jako limitu funkce dvou parametrů. Znamená to tedy, že funkční hodnota se limitě musí blížit \uv{ze všech směrů}.

\begin{prolog}
?-	make_test_real_pairs(Pairs),
	make_test_complex_functions(P, Functions),
	print_validated_formula(
		'complex_lim',
		declare([variable([X0, Y0], ['x_0', 'y_0'], Pairs), function(F, '\\func{f}', Functions)],
			equal(
				lim(Z, 'z', complex, X0 + Y0 * imag, apply(F, [P], [Z])),
				lim([X, Y], ['x', 'y'], [real, real], [X0, Y0], apply(F, [P], [X + Y * imag]))
			)
		)
	).
\end{prolog}
\eeq{complex_lim}

Derivaci komplexní funkce definujeme obdobně jako derivaci reálné funkce pomocí vztahu~\eqref{eq:complex_derivative}.

\begin{prolog}
?-	make_test_numbers(Numbers),
	make_test_complex_functions(P, Functions),
	print_validated_formula(
		"complex_derivative",
		declare([function(F, "\\func{f}", Functions), variable(Z, 'z', Numbers)],
			equal(
				derivative(Z, apply(F, [P], [Z])),
				lim(DZ, "\\mathrm{d}z", complex, 0, (apply(F, [P], [Z + DZ]) - apply(F, [P], [Z])) / DZ)
			)
		)
	).
\end{prolog}
\eeq{complex_derivative}

Zabývejme se otázkou, kdy tato derivace existuje. Nejdříve rozepišme definici derivace~\eqref{eq:complex_derivative} na podíl diferenciálů. Zavedeme tedy označení \(\func{f}(z + \mathrm{d}z) - \func{f}(z) = \mathrm{d}u + \mathrm{d}v \cdot \imag\) a~\(\mathrm{d}z = \mathrm{d}x + \mathrm{d}y \cdot \imag\):

\begin{prolog}
?-	make_test_real_binary_functions(FX, FY, Functions),
	make_test_real_pairs(Pairs),
	print_validated_formula(
		"complex_derivative_conditions_1",
		declare([
			variable([X, Y], ["x", "y"], Pairs),
			function(U, "u", Functions),
			function(V, "v", Functions),
			substitution(DU, "\\mathrm{d}u", apply(U, [FX, FY], [X + DX, Y + DY]) - apply(U, [FX, FY], [X, Y])),
			substitution(DV, "\\mathrm{d}v", apply(V, [FX, FY], [X + DX, Y + DY]) - apply(V, [FX, FY], [X, Y]))
		],
			equal([
				lim(
					[DX, DY], ["\\mathrm{d}x", "\\mathrm{d}y"], [real, real], [0, 0],
					(DU + DV * imag) / (DX + DY * imag)
				),
				lim([DX, DY], ["\\mathrm{d}x", "\\mathrm{d}y"], [real, real], [0, 0],
					((DU + DV * imag) / (DX + DY * imag)) * ((DX - DY * imag) / (DX - DY * imag))	
				),
				linebreak,
				lim(
					[DX, DY], ["\\mathrm{d}x", "\\mathrm{d}y"], [real, real], [0, 0],
					(DU * DX + DV * DY + (DV * DX - DU * DY) * imag) / (DX^2 + DY^2)
				)
			])
		)
	).
\end{prolog}
\eeq{complex_derivative_conditions_1}

Uvědomme si, že limita je definována pro 

\section{Funkce sinus a~kosinus}
