\chapter{Komplexní čísla}


\begin{prolog}
:- ensure_loaded("../equations/formula").
:- ensure_loaded("../equations/truth_table").
:- ensure_loaded("../equations/draw_1d_function").

make_test_numbers([-1, 0, 0.5, 1, 2, 3, 1.5, complex(0, 1), complex(0, -1), complex(1, 1)]).	
\end{prolog}

V~sekci \ref{sec:realna_cisla} věnované reálným číslům jsme větou~\eqref{eq:sqrt_def} definovali odmocninu kladných čísel. Důvodem tohoto omezení je fakt, že v oboru reálných čísel neexistují sudé odmocniny ze záporných čísel. Například rovnice \(x^2 = -1\) nemá v~oboru reálných čísel řešení, neexistuje proto \(\sqrt{-1}\). Abychom mohli takovéto rovnice řešit, tak rovnicí~\eqref{eq:imag_def} zavedeme imaginární jednotku \(\imag\).

\begin{fact}
\begin{prolog}
?-	print_validated_formula(
		'imag_def',
		equal(imag^2, -1)
	).
\end{prolog}
\eeq{imag_def}
\end{fact}

Uvažujme dvě reálná čísla \(a\) a~\(b\), z~nichž alespoň jedno je nenulové. Chtějme nalézt takovou kombinaci těchto čísel, aby platilo:

\begin{prolog}
?-	make_test_numbers(Numbers),
	print_validated_formula(
		'complex_units_ortogonality',
		declare_variable(A, 'a', Numbers,
			declare_variable(B, 'b', Numbers,
				proof([
					equal(A * 1 + B * imag, 0)
				],
				[
					equal(A * 1, -B * imag),
					equal(A^2 * 1^2, B^2 * imag^2),
					equal(A^2, -B^2),
					equal(A^2 + B^2, 0)
				])
			)
		)
	).
\end{prolog}
\eeq{complex_units_ortogonality}

Vidíme, že pro nenulová čísla tato rovnice nemůže být splněna. Reálná jednotka \(1\) a~imaginární jednotka \(\imag\) jsou proto navzájem ortogonální - jednu není možné vyjádřit pomocí druhé. Tyto jednotky si tedy můžeme představit jako vzájemně kolmé jednotkové osy ve dvojrozměrném prostoru. Každý bod v~tomto prostoru můžeme popsat jako lineární kombinaci těchto jednotek. Proto tuto lineární kombinaci \(a + b \cdot \imag\) nazveme komplexním číslem. Množinu všech komplexních čísel označíme \(\complex\).

\begin{fact}
\begin{prolog}
?-	make_test_numbers(Numbers),
	print_validated_formula(
		'complex_def',
		forall_in([A, B], ['a', 'b'], real_numbers, Numbers,
			in(A + B * imag, complex_numbers)
		)
	).
\end{prolog}
\eeq{complex_def}
\end{fact}

