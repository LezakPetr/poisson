\chapter{Derivace základních funkcí}
\label{ap:derivace_zakladnich_funkci}

\begin{prolog}
:- ensure_loaded("../equations/formula").

make_test_real_numbers([-2, -1, 0, 1, 1.5, 2]).
make_test_natural_numbers([1, 2, 3, 4, 5]).
make_test_real_functions(FX, [FX, FX^2, FX + 1, (FX + 1) * FX^5]).
\end{prolog}


V~této příloze jsou sespsány derivace základních funkcí, které byly odvozeny v~sekci~\ref{sec:derivace_zakladnich_funkci}.

\begin{prolog}
?-	make_test_real_numbers(Numbers),
	print_validated_formula(
		"ap_derivative_constant",
		declare(
			[
				variable(X, "x", Numbers),
				variable(K, "k", Numbers)
			],
			equal(derivative(X, K), 0)
		)
	).
\end{prolog}
\eeq{ap_derivative_constant}

\begin{prolog}
?-	make_test_real_numbers(Numbers),
	print_validated_formula(
		"ap_derivative_variable",
		declare(
			[
				variable(X, "x", Numbers)
			],
			equal(derivative(X, X), 1)
		)
	).
\end{prolog}
\eeq{ap_derivative_variable}

\begin{prolog}
?-	make_test_real_numbers(Numbers),
	make_test_real_functions(FX, FunctionsF),
	make_test_real_functions(GX, FunctionsG),
	print_validated_formula(
		"ap_derivative_plus_minus",
		declare(
			[
				variable(X, "x", Numbers),
				function(F, "f", FunctionsF),
				function(G, "g", FunctionsG),
				plus_minus(PM)
			],
			equal(
				derivative(X, plus_minus(apply(F, [FX], [X]), apply(G, [GX], [X]), PM)),
				plus_minus(derivative(X, apply(F, [FX], [X])), derivative(X, apply(G, [GX], [X])), PM)
			)
		)
	).
\end{prolog}
\eeq{ap_derivative_plus_minus}

\begin{prolog}
?-	make_test_real_numbers(Numbers),
	make_test_real_functions(FX, FunctionsF),
	make_test_real_functions(GX, FunctionsG),
	print_validated_formula(
		"ap_derivative_multiply",
		declare(
			[
				variable(X, "x", Numbers),
				function(F, "f", FunctionsF),
				function(G, "g", FunctionsG)
			],
			equal(
				derivative(X, apply(F, [FX], [X]) * apply(G, [GX], [X])),
				derivative(X, apply(F, [FX], [X])) * apply(G, [GX], [X]) + 
				apply(F, [FX], [X]) * derivative(X, apply(G, [GX], [X]))
			)
		)
	).
\end{prolog}
\eeq{ap_derivative_multiply}

\begin{prolog}
?-	make_test_real_numbers(Numbers),
	make_test_real_functions(FX, FunctionsF),
	print_validated_formula(
		"ap_derivative_multiply_by_constant",
		declare(
			[
				variable(X, "x", Numbers),
				function(F, "f", FunctionsF),
				variable(K, "k", Numbers)
			],
			equal(
				derivative(X, K * apply(F, [FX], [X])),
				K * derivative(X, apply(F, [FX], [X]))
			)
		)
	).
\end{prolog}
\eeq{ap_derivative_multiply_by_constant}

\begin{prolog}
?-	make_test_real_numbers(Numbers),
	make_test_natural_numbers(NaturalNumbers),
	print_validated_formula(
		"ap_derivative_power_by_natural",
		declare(
			[
				variable(X, "x", Numbers)
			],
			forall_in(N, "n", natural_numbers, NaturalNumbers, 
				equal(
					derivative(X, X^N),
					N * X^(N - 1)
				)
			)
		)
	).
\end{prolog}
\eeq{ap_derivative_power_by_natural}

\begin{prolog}
?-	make_test_real_numbers(Numbers),
	print_validated_formula(
		"ap_derivative_power_by_real",
		declare(
			[
				variable(R, "r", Numbers)
			],
			forall_in(X, "x", positive_real_numbers, Numbers, 
				equal(
					derivative(X, X^R),
					R * X^(R - 1)
				)
			)
		)
	).
\end{prolog}
\eeq{ap_derivative_power_by_real}

\begin{prolog}
?-	make_test_real_numbers(Numbers),
	print_validated_formula(
		"ap_derivative_exp",
		declare(
			[
				variable(X, "x", Numbers)
			],
			equal(
				derivative(X, e^X),
				e^X
			)
		)
	).
\end{prolog}
\eeq{ap_derivative_exp}

\begin{prolog}
?-	make_test_real_numbers(Numbers),
	print_validated_formula(
		"ap_derivative_general_power",
		declare(
			[
				variable(X, "x", Numbers)
			],
			forall_in(A, "a", positive_real_numbers, Numbers, 
				equal(
					derivative(X, A^X),
					ln(A) * A^X
				)
			)
		)
	).
\end{prolog}
\eeq{ap_derivative_general_power}

\begin{prolog}
?-	make_test_real_numbers(Numbers),
	print_validated_formula(
		"ap_derivative_ln",
		forall_in(X, "x", positive_real_numbers, Numbers, 
			equal(
				derivative(X, ln(X)),
				1 / X
			)
		)
	).
\end{prolog}
\eeq{ap_derivative_ln}

\begin{equation}
\frac{\partial}{\partial x} \mathrm{f} (\mathrm{g}_1(x, y, ...), \mathrm{g}_2(x, y, ...), ..., \mathrm{g}_n(x, y, ...)) = \sum_{i=1}^n \frac{\partial \mathrm{f}}{\partial g_i}(\mathrm{g}(x)) \cdot \frac{\partial \mathrm{g}_i}{\partial x}
\end{equation}
