\chapter{Derivace základních funkcí}
\label{ap:derivace_zakladnich_funkci}

V~této příloze jsou sespsány derivace základních funkcí, které byly odvozeny v~sekci~\ref{sec:derivace_zakladnich_funkci}.

\begin{equation}
\frac{\partial k}{\partial x} = 0
\end{equation}

\begin{equation}
\frac{\partial}{\partial x} x = 1
\end{equation}

\begin{equation}
\frac{\partial}{\partial x} (\mathrm{f}(x) \pm \mathrm{g}(x)) = \frac{\partial \mathrm{f}}{\partial x} \pm \frac{\partial \mathrm{g}}{\partial x}
\end{equation}

\begin{equation}
\frac{\partial}{\partial x} (\mathrm{f}(x) \cdot \mathrm{g}(x)) = \mathrm{g}(x) \cdot \frac{\partial \mathrm{f}}{\partial x} + \mathrm{f}(x) \cdot \frac{\partial \mathrm{g}}{\partial x}
\end{equation}

\begin{equation}
\frac{\partial}{\partial x} (k \cdot \mathrm{f}(x)) = k \cdot \frac{\partial \mathrm{f}}{\partial x}
\end{equation}

\begin{equation}
\frac{\partial}{\partial x} x^n = n \cdot x^{n-1}; n \in \mathbb{N}
\end{equation}

\begin{equation}
\frac{\partial}{\partial x} x^r = r \cdot x^{r-1}; x > 0
\end{equation}

\begin{equation}
\frac{\partial}{\partial x} e^x = e^x
\end{equation}

\begin{equation}
\frac{\partial}{\partial x} a^x = \ln a \cdot a^x
\end{equation}

\begin{equation}
\frac{\partial}{\partial x} \ln x = \frac{1}{x}; x > 0
\end{equation}

\begin{equation}
\frac{\partial}{\partial x} \mathrm{f} (\mathrm{g}_1(x, y, ...), \mathrm{g}_2(x, y, ...), ..., \mathrm{g}_n(x, y, ...)) = \sum_{i=1}^n \frac{\partial \mathrm{f}}{\partial g_i}(\mathrm{g}(x)) \cdot \frac{\partial \mathrm{g}_i}{\partial x}
\end{equation}
