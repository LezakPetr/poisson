\chapter{Popis veličin v prostoru}

\section{Změna skalárního pole v~daném směru, gradient}

Mějme skalární funkci \(\varphi(P_1, P_2, ..., P_n)\) definovanou obecně v~křivočarých souřadnicích \(P_i\). Chceme-li vypočítat její změnu v~nekonečně blízkém bodě \((P_1 + \mathrm{d}P_1, P_2 + \mathrm{d}P_2, ..., P_n + \mathrm{d}P_n)\), pak
můžeme použít vztah pro totální diferenciál funkce:

\begin{equation}
\begin{split}
\mathrm{d} \varphi = \varphi(P_1 + \mathrm{d}P_1, P_2 + \mathrm{d}P_2, ..., P_n + \mathrm{d}P_n) - \varphi(P_1, P_2, ..., P_n) = \\
\mathrm{d}P_1 \cdot \frac{\partial \varphi}{\partial P_1} + \mathrm{d}P_1 \cdot \frac{\partial \varphi}{\partial P_2} +
... + \mathrm{d}P_n \cdot \frac{\partial \varphi}{\partial P_n} = \\
\sum_{i=1}^n \mathrm{d}P_i \cdot \frac{\partial \varphi}{\partial P_i}
\end{split}
\end{equation}

Kovariantní vektor \(v_i = \frac{\partial \varphi}{\partial P_i}\) nazveme gradientem a~označíme \(\kovarvect{v} = \grad{\varphi}\). Změnu funkční hodnoty pak vypočítáme:

\begin{equation}
\label{eq:zmena_pole_gradient}
\begin{split}
\mathrm{d}\varphi = (\mathrm{d}P_1, \mathrm{d}P_2, ..., \mathrm{d}P_n) \cdot \grad{\varphi} 
\end{split}
\end{equation}

Dále budeme uvažovat pouze kartézský souřadný systém. V~něm víme, že skalární součin dvou vektorů nabývá svého maxima, pokud
mají oba vektory stejný směr. Gradient v~kartézském souřadném systému proto má směr nejvyššího růstu funkce \(\varphi\).

% TODO kolmá konstantní rovina


\section{Operátory}

Dále se seznámíme s operátory, které působí na pole. Již známe operátory divergence a rotace, oba působí na vektorová pole. Divergence vektorového pole
je skalární pole, které má v~každém bodě hodnotu celkového výtoku z~infinitezimálního objemu okolo daného bodu. Rotace je vektorového pole je vektorové
pole, které v každém bodě definuje cirkulaci z nekonečně malých smyček.

Další operátor je gradient. Gradient skalárního pole \(\varphi\) je vektorové pole \(\vect{v}\), které má v~každém bodě směr nejvyššího přírůstku skalárního pole, a velikost rovnou derivaci pole v tomto směru. Je definován vztahem
\[
\vect{v} = \grad \ \varphi = \left(\frac{\partial \varphi}{x}, \frac{\partial \varphi}{y}, \frac{\partial \varphi}{z}\right)
\].

Laplaceův operátor \(\Delta\) přiřazuje skalárnímu poli \(\varphi\) skalární pole \(k\) podle vztahu \(\Delta \varphi = \diverg \ \grad \ \varphi = \frac{\partial^2 \varphi}{\partial x^2} + \frac{\partial^2 \varphi}{\partial y^2} + \frac{\partial^2 \varphi}{\partial z^2}\). Operátor může působit i na vektorové pole, v tom případě působí na každou složku zvlášť. Tedy \(\Delta \ \vect{v} = \left(\Delta \ v_x, \Delta \ v_y, \Delta \ v_z \right) = \left(\frac{\partial^2 v_x}{\partial x^2} + \frac{\partial^2 v_x}{\partial y^2} + \frac{\partial^2 v_x}{\partial z^2},
\frac{\partial^2 v_y}{\partial x^2} + \frac{\partial^2 v_y}{\partial y^2} + \frac{\partial^2 v_y}{\partial z^2},
\frac{\partial^2 v_z}{\partial x^2} + \frac{\partial^2 v_z}{\partial y^2} + \frac{\partial^2 v_z}{\partial z^2} \right)\).

Vektorový Laplaceův operátor je také možné rozepsat na součet vírového a zřídlového pole jak ukazuje rovnice \eqref{eq:vect_laplace_rozepsany}.

\begin{equation}
\label{eq:vect_laplace_rozepsany}
\Delta \ \vect{v} = \grad \ \diverg \ \vect{v} - \rot \ \rot \vect{v} 
\end{equation}

\subsection{Vztahy mezi operátory}

Gradient, divergence, rotace i aplikace laplaceova operátoru jsou lineární operace. Nechť \(\alpha\) a \(\beta\) jsou konstanty, \(\varphi\) a \(rho\) skalární pole a \(u\) a \(v\) vektorová pole. Pak platí tyto vztahy:
\[
\grad(\alpha \cdot \varphi + \beta \cdot \rho) = \alpha \cdot \grad \ \varphi + \beta \cdot \grad \ \rho
\]
\[
\diverg(\alpha \cdot \vect{u} + \beta \cdot \vect{v}) = \alpha \cdot \diverg \ \vect{u} + \beta \cdot \diverg \ \vect{v}
\]
\[
\rot(\alpha \cdot \vect{u} + \beta \cdot \vect{v}) = \alpha \cdot \rot \ \vect{u} + \beta \cdot \rot \ \vect{v}
\]

Tyto vztahy lze jednoduše ověřit dosazením do jejich definic.

Dále jsou důležité následující vztahy mezi operátory.

\begin{equation}
\label{eq:div_rot}
\diverg \ \rot \ \vect{u} = 0
\end{equation}

\begin{equation}
\label{eq:rot_grad}
\rot \ \grad \ \varphi = \vect{0}
\end{equation}

\section{Pole zřídlová a vírová}

Mějme vektorové pole \(\vect{u}\) v prostoru, které má v uvažované oblasti definované derivace prvního řádu. Toto pole má obecně nenulovou divergenci a rotaci.
Pokusíme se toto pole rozložit na součet dvou polí - pole \(\vect{v}\) s nulovou rotací a pole \(\vect{w}\) s nulovou divergencí.

\begin{equation}
\vect{u} = \vect{v} + \vect{w}
\end{equation}

\begin{equation}
\rot \ \vect{v} = \vect{0}
\end{equation}

\begin{equation}
\diverg \ \vect{w} = 0
\end{equation}

To můžeme udělat tak, že stanovíme \(\vect{v} = \grad \ \varphi\). Podmínka nulové rotace je splněna, protože \(\rot \ \vect{v} = \rot \ \grad \ \varphi = \vect{0}\).
Potenciál \(\varphi\) stanovíme tak, aby \(\diverg \ \vect{u} = \diverg \ \vect{v} = \diverg \ \grad \ \varphi\). To je Poissonova rovnice bez okrajových podmínek, která má řešení popsané vztahem \eqref{eq:potencial_v_nekonecnem_prostoru}.

Pole \(\vect{w}\) je tedy \(\vect{w} = \vect{u} - \vect{v}\). Vyšetřeme jeho divergenci. \(\diverg \ \vect{w} = \diverg \ \vect{u} - \diverg \ \vect{v} = \diverg \ \vect{u} - \diverg \ \vect{u} = 0\).

Vidíme, že pole \(u\) je tedy možné rozložit na součet pole s nulovou rotací a pole s nulovou divergencí. Jak uvidíme v sekci \ref{sec:pole_virova}, tak pole
s nulovou divergencí je možné zapsat jako rotaci vektorového potenciálu. Pole \(\vect{u}\) je proto možné rozložit na součet gradientu skalárního potenciálu
a rotace vektorového potenciálu, jak popisuje rovnice \eqref{eq:rozklad_grad_rot}.

\begin{equation}
\label{eq:rozklad_grad_rot}
\vect{u} = \grad \ \varphi + \rot \vect{\psi}
\end{equation}

Podívejme se na obě složky podrobněji.

\subsection{Pole zřídlová}

Mějme pole \(v\), které má v uvažované oblasti nulovou rotaci. Pak podle Stokesovy věty cirkulace vektoru \(v\) všemi uzavřenými smyčkami v uvažované
oblasti je nulová. Mějme v oblasti 2 body A a B. Dále mějme 2 křivky \(\Gamma_1\) a \(\Gamma_2\) které obě začínají v bodě A a končí v bodě B. Pak musí platit rovnost \eqref{eq:potencial_1}. Plyne to z faktu, že otočíme-li jednu z křivek, pak jejich spojením získáme smyčku, která musí mít nulovou cirkulaci.

\begin{equation}
\label{eq:potencial_1}
\int_{\Gamma_1} \vect{v} \cdot d\vect{l} = \int_{\Gamma_2} \vect{v} \cdot d\vect{l}
\end{equation}

To ale znamená, že křivkový integrál potenciálního pole závisí pouze na počátečním a koncovém bodu, ne na tvaru křivky. Dále platí, že křivkové integrály po navazujících křivkách se sčítají. To nám umožňuje zavést skalární pole - potenciál \(\varphi\) splňující podmínku \eqref{eq:potencial_2}. Takováto pole nazýváme zřídlová, potenciální nebo také bezvírová.

\begin{equation}
\label{eq:potencial_2}
\int_{AB} \vect{v} \cdot d\vect{l} = \varphi(\vect{B}) - \varphi(\vect{A})
\end{equation}

Vyšetřeme, jaký je vztah mezi potenciálem \(\varphi\) a polem \(\vect{v}\). Nechť bod \(\vect{B}\) leží v infinitizimálně malé vzdálenosti \(dx\) od bodu
\(\vect{A}\) ve směru osy \(x\). Křivka \(AB\) bude úsečka mezi body \(\vect{A}\) a \(\vect{B}\), má tedy směr osy \(x\). Vztah \eqref{eq:potencial_2} proto lze přepsat do tvaru \eqref{eq:potencial_3}.

\begin{equation}
\label{eq:potencial_3}
\begin{split}
v_x \cdot dx = \varphi(\vect{A} + (dx, 0, 0)) - \varphi(\vect{A}) \\
v_x = \frac{\varphi(\vect{A} + (dx, 0, 0)) - \varphi(\vect{A})}{dx}
\end{split}
\end{equation}

Pro \(dx \rightarrow 0\) získáme vztah \eqref{eq:potencial_4}.

\begin{equation}
\label{eq:potencial_4}
v_x = \frac{\partial \varphi}{\partial x}
\end{equation}

Stejně budeme postupovat pro osy \(y\) a \(z\). Získáme tak vztah \eqref{eq:potencial_5}.

\begin{equation}
\label{eq:potencial_5}
\vect{v} = \left(\frac{\partial \varphi}{\partial x}, \frac{\partial \varphi}{\partial y}, \frac{\partial \varphi}{\partial z}\right) = \grad \ \varphi
\end{equation}

Vidíme tedy, že pole s nulovou rotací lze zapsat jako gradient skalárního potenciálu. Ověřme, že nulová rotace je podmínkou nutnou a postačující pro existenci
potenciálu. Snažíme se tedy dokázat, že pro dané vektorové pole \(v\) existuje potenciál \(\varphi\) takový, že \(\vect{v} = \grad \ \varphi\) tehdy a jen tehdy
pokud \(\rot \ \vect{v} = \vect{0}\).

Podmínku \(\rot \ \vect{v} = \vect{0}\) můžeme rozepsat na \(\frac{\partial v_x}{\partial y} = \frac{\partial v_y}{\partial x}\), \(\frac{\partial v_x}{\partial z} = \frac{\partial v_z}{\partial x}\) a \(\frac{\partial v_y}{\partial z} = \frac{\partial v_z}{\partial y}\).

Že se jedná o podmínku nutnou ověříme snadno. Z definice gradientu plyne \(v_x = \frac{\partial \varphi}{\partial x}\) a \(v_y = \frac{\partial \varphi}{\partial y}\). První rovnici zderivujeme podle \(y\) a získáme \(\frac{\partial v_x}{\partial y} = \frac{\partial^2 \varphi}{\partial x y}\). Druhou rovnici
zderivujeme podle \(x\) a získáme \(\frac{\partial v_y}{\partial x} = \frac{\partial^2 \varphi}{\partial y x}\). Srovnáme-li pravé strany těchto rovnic, pak
díky nezávislosti pořadí derivování jsou shodné. Proto musí platit \(\frac{\partial v_x}{\partial y} = \frac{\partial v_y}{\partial x}\). Obdobně můžeme
postupovat pro ostatní kombinace x-z a y-z a získáme tak zbylé podmínky.

Ověřit, že se jedná o podmínku postačující je poněkud složitější. Rovnice \eqref{eq:potencial_2} nám dává návod, jak potenciál určit. Zvolíme si pevný bod
v oblasti \(A\), v něm zvolíme potenciál \(\varphi_0\). Od tohoto bodu vedeme libovolnou křivku do bodu \(B\), kde hledáme potenciál. Potenciál v něm
pak spočítáme podle vztahu \eqref{eq:potencial_6}. Potenciál \(\varphi\) je tak až na konstantu \(\varphi_0\) plně určen polem \(v\).

\begin{equation}
\label{eq:potencial_6}
\varphi(\vect{B}) = \varphi_0 + \int_{AB} \vect{v} \cdot d\vect{l}
\end{equation}

Abychom ověřili, že tímto postupem skutečně dostaneme potenciál pole \(v\), tak zvolíme takovou křivku \(AB\), která nám umožní snadno spočítat gradient
potenciálu. Za bod \(A\) zvolíme počátek souřadnicového systému. Křivku \(AB\)) vedeme z počátku po ose \(x\) do vzdálenosti \(B_x\), pak ve směru osy
\(y\) do vzdálenosti \(B_y\) a nakonec ve směru osy \(z\) do vzdálenosti \(B_z\), tedy do bodu \(B\). Potenciál je tak určen vztahem \eqref{eq:potencial_7}.

\begin{equation}
\label{eq:potencial_7}
\varphi(\vect{B}) = \varphi_0 + \int_0^{B_x} v_x(x, 0, 0) dx + \int_0^{B_y} v_y(B_x, y, 0) dy  + \int_0^{B_z} v_z(B_x, B_y, z) dz
\end{equation}

Spočítejme nyní parciální derivace potenciálu \(\varphi\). Začněme derivací podle \(z\), přesněji \(B_z\). Na \(B_z\) závisí pouze poslední integrál
integrující podle \(dz\), \(B_z\) se vyskytuje v horní integrační mezi. Derivováním tohoto integrálu proto získáme vztah \eqref{eq:potencial_vz}.

\begin{equation}
\label{eq:potencial_vz}
\frac{\partial \varphi}{B_z} = v_z(B_x, B_y, B_z)
\end{equation}

Pokračujme derivací podle \(y\). \(B_y\) se vyskytuje v horní mezi druhého integrálu a dále ve třetím integrálu. Získáme tak vztah \label{eq:potencial_vy}.
Při jeho úpravě využijeme předpoklad \(\frac{\partial v_y}{\partial z} = \frac{\partial v_z}{\partial y}\).

\begin{equation}
\label{eq:potencial_vy}
\begin{split}
\frac{\partial \varphi}{B_y} = v_y(B_x, B_y, 0) + \int_0^{B_z} \frac{\partial v_z(B_x, B_y, z)}{B_y} dz = \\
v_y(B_x, B_y, 0) + \int_0^{B_z} \frac{\partial v_y(B_x, B_y, z)}{B_z} dz = \\
v_y(B_x, B_y, 0) + v_y(B_x, B_y, B_z) - v_y(B_x, B_y, 0) dz = \\
v_y(B_x, B_y, B_z)
\end{split}
\end{equation}

Nakonec spočítáme derivací podle \(x\). \(B_x\) se vyskytuje v horní mezi prvního integrálu a dále ve druhém a třetím integrálu. Získáme tak vztah \eqref{eq:potencial_vx}. Při jeho úpravě využijeme předpoklady \(\frac{\partial v_x}{\partial z} = \frac{\partial v_z}{\partial x}\) a
\(\frac{\partial v_x}{\partial y} = \frac{\partial v_y}{\partial x}\).

\begin{equation}
\label{eq:potencial_vx}
\begin{split}
\frac{\partial \varphi}{B_x} = v_x(B_x, 0, 0) + \int_0^{B_y} \frac{\partial v_y(B_x, y, 0)}{B_x} dy + \int_0^{B_z} \frac{\partial v_z(B_x, B_y, z)}{B_x} dz = \\
v_x(B_x, 0, 0) + \int_0^{B_y} \frac{\partial v_x(B_x, y, 0)}{B_y} dy + \int_0^{B_z} \frac{\partial v_x(B_x, B_y, z)}{B_z} dz = \\
v_x(B_x, 0, 0) + v_x(B_x, B_y, 0) - v_x(B_x, 0, 0) + v_x(B_x, B_y, B_z) - v_x(B_x, B_y, 0) = \\
v_x(B_x, B_y, B_z)
\end{split}
\end{equation}

Vidíme, že všechny parciální derivace potenciálu odpovídají složkám gradientu. Dokázali jsme tak, že má-li pole \(v\) nulovou rotaci, můžeme zavést jeho
potenciál vztahem \eqref{eq:potencial_7} a díky vztahu \eqref{eq:potencial_1} obecně vztahem \eqref{eq:potencial_6}.

\subsubsection{Potenciál jednoparametrického pole}
\label{sec:potencial_jednoparametrickeho_pole}
Rozeberme jeden speciální případ, který se často vyskytuje.

\begin{fact}
Mějme ne nutně kartézský souřadný systém s~body určenými souřadnicemi \(P\). Pokud pole \(\kovarvect{v}(P)\) má pouze jednu nenulovou složku kovariantních souřadnic a~pokud tato složka závisí pouze na této souřadnici, pak je \(\kovarvect{v}(P)\) potenciální pole. 
\end{fact}

Dokažme to. Beze ztráty obecnosti uvažujme, že první složka pole \(\kovarvect{v}(P)\) je nenulová, ostatní jsou nulové. Můžeme tedy zapsat:

\begin{equation}
(v_1(P_1), 0, 0, ..., 0) = \grad \varphi = \left(\frac{\partial \varphi}{\partial P_1}, 0, 0, ...,  0 \right)
\end{equation}

Pro první složku tedy dostáváme obyčejnou diferenciální rovnici, která má řešení

\begin{equation}
\begin{split}
v_1(P_1) = \frac{\partial \varphi}{\partial P_1} \\
\varphi(P_1) = \int v_1(P_1) \ \mathrm{d}P_1 + C
\end{split}
\end{equation}

Rovnost ostatních složek je zajištěna volbou potenciálu jako funkce první složky. Vidíme, že vhodnou volbou souřadného systému můžeme potenciál snadno vypočítat.

\subsection{Pole vírová}
\label{sec:pole_virova}

Vektorová pole \(\vect{v}\) mající v celé uvažované oblasti nulovou divergenci nazýváme pole vírová. Takováto pole můžeme zapsat ve formě 
\eqref{eq:virova_pole_definice}. Veličinu \(\vect{\psi}\) nazýváme vektorovým potenciálem pole \(\vect{v}\).

\begin{equation}
\label{eq:virova_pole_definice}
\vect{v} = \rot \ \vect{\psi}
\end{equation}

Ze vztahu \eqref{eq:div_rot} je zřejmé, že nulová divergence pole \(\vect{v}\) je podmínkou nutnou pro existenci vektorového potenciálu. Že se jedná o podmínku postačující a jak potenciál určit uvidíme dále.

Z rovnice \eqref{eq:virova_pole_nejednoznacnost} je vidět, že vektorový potenciál není polem \(\vect{v}\) určen jednoznačně.
Můžeme k němu přičíst libovolné pole s nulovou rotací, tedy libovolné potenciální pole, a rovnice \eqref{eq:virova_pole_definice}
zůstane v platnosti.

\begin{equation}
\label{eq:virova_pole_nejednoznacnost}
\rot \left(\vect{\psi} + \grad \varphi \right) = \rot \ \vect{\psi} + \rot \ \grad \varphi =
\rot \ \vect{\psi} + \vect{0} = \rot \ \vect{\psi}
\end{equation}

Speciálním případem je situace, kdy \(\varphi\) zvolíme tak, že \(\diverg \ \vect{\psi} + \diverg \ \grad \ \varphi = 0\). To bude splněno pokud
\(-\diverg \vect{\psi} = \diverg \grad \varphi = 0\). Pravá strana rovnice představuje Poissonovu rovnici bez okrajových podmínek.
Její řešení je uvedeno dále, ale nyní je podstatná pouze jeho existence. Důsledkem tedy je, že existuje-li pole \(\vect{\psi}\) řešící
rovnici \eqref{eq:virova_pole_definice}, pak existuje \(\vect{\psi}\) takové, že má nulovou divergenci, tedy vírové pole.

Vyjdeme z rovnice \eqref{eq:virova_pole_definice}. Beze ztráty na obecnosti můžeme potenciál \(\varphi\) považovat za vírové pole.
Zaveďme tedy potenciál \(\vect{\rho}\) vztahem \(\vect{\psi} = \rot \ \vect{\rho}\). Abychom určili potenciál \(\vect{\rho}\), tak tedy
potřebujeme vyřešit rovnici \eqref{eq:virova_pole_potencial_1}.

\begin{equation}
\label{eq:virova_pole_potencial_1}
\vect{v} = \rot \ \rot \ \vect{\rho}
\end{equation}

Jak jsme viděli výše, tak můžeme zavést podmínku \(\diverg \ \vect{\rho} = 0\). Pak ale můžeme k rovnici \eqref{eq:virova_pole_potencial_1}
přičíst \(\grad \ \diverg \ \vect{\rho}\), který je nulový. Dále ještě rovnici invertujeme. Získáme tak rovnici \eqref{eq:virova_pole_potencial_2}.

\begin{equation}
\label{eq:virova_pole_potencial_2}
-\vect{v} = \grad \ \diverg \ \vect{\rho} - \rot \ \rot \ \vect{\rho}
\end{equation}

Pravou stranu rovnice můžeme přepsat podle vztahu \eqref{eq:vect_laplace_rozepsany}, získáme tak rovnici \eqref{eq:virova_pole_potencial_3}.

\begin{equation}
\label{eq:virova_pole_potencial_3}
-\vect{v} = \Delta \ \vect{\rho}
\end{equation}

To je vektorová Poissonova rovnice bez okrajových podmínek. Jak uvidíme dále, tak tato rovnice má řešení. Proto vírová pole \(v\) lze vyjádřit
pomocí pomocí rovnice \eqref{eq:virova_pole_definice} a to tak, že vyřešíme rovnici \eqref{eq:virova_pole_potencial_3} a dopočítáme \(\vect{\psi} = \rot \ \vect{\rho}\).

Nutnou a~postačující podmínkou pro vírové pole je proto:

\begin{equation}
\diverg \ \vect{v} = 0
\end{equation}

Ukážeme si, jak tuto podmínku zapsat v~integrálním tvaru. Obě strany rovnice budeme integrovat přes libovolný objem \(V\):

\begin{equation}
\int_{V} \diverg \ \ \vect{v} \ \mathrm{d}V = \int_{V} 0 \ \mathrm{d}V = 0
\end{equation}

Na levou stranu rovnice aplikujeme Gaussův teorém:

\begin{equation}
\oint_{\partial V} \vect{v} \ \mathrm{d}\vect{s} = 0
\end{equation}

Protože objem \(V\) byl zvolen libovolně, tak i~hranice \(\partial V\) je libovolná. Proto vztah můžeme zobecnit:

\begin{equation}
\oint_{S} \vect{v} \ \mathrm{d}\vect{s} = 0
\end{equation}

Proto se polím říká bezzřídlová, protože tato pole neobsahují žádné zdroje - zřídla.

\begin{fact}
Vírová neboli bezzřídlová pole jsou pole, která lze zapsat pomocí vztahu

\begin{equation}
\vect{v} = \rot \ \vect{\psi}
\end{equation}

Nutnou a~postačující podmínkou je platnost rovnice

\begin{equation}
\diverg \ \vect{v} = 0
\end{equation} 

pro libovolný bod neboli

\begin{equation}
\oint_{S} \vect{v} \ \mathrm{d}\vect{s} = 0
\end{equation}

pro libovolnou uzavřenou plochu \(S\).
\end{fact}
