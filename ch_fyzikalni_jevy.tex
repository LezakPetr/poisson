\chapter{Fyzikální jevy}

Poissonova rovnice umožňuje řešit celou řadu různých fyzikálních jevů. Níže si některé z~nich popíšeme. V~některých případech
je ale nutné fyzikální rovnici napřed upravit na Poissonovu rovnici.

\section{Tepelný tok}

Mějme izotropní tepelně vodivý materiál s~tepelnou vodivostí \(\lambda\). 
Máme-li v materiálu 2 plochy o~obsahu \(S\) ve vzdálenosti \(v\) s~teplotním rozdílem \(\mathrm{d}t\), pak mezi nimi bude proudit tepelný tok 

\begin{equation}
\label{eq:tepelny_tok_1}
P = -\lambda S \frac{\mathrm{d}t}{v}
\end{equation}

Záporné znaménko značí, že směr proudu tepla je opačný než směr růstu
teploty - teplo proudí z~místa s~vyšší teplotou do místa s~nižší teplotou. 

Zapišme tento vztah vektorově. 
Začneme tím, že nahradíme vzdálenost \(v\) vektorem nekonečně malého posunutí \(\mathrm{d}\vect{v}\). Dále podle vztahu \eqref{eq:zmena_pole_gradient} platí \(\mathrm{d}t = \mathrm{t}(x + \mathrm{d}v_x, y + \mathrm{d}v_y, z + \mathrm{d}v_z) - t(x, y, z) = \mathrm{d}\vect{v} \cdot \grad \ \mathrm{t}\):

\begin{equation}
P = -\lambda S \frac{\mathrm{d}\vect{v} \cdot \grad \ \mathrm{t}}{|\mathrm{d}\vect{v}|}
\end{equation}

Vidíme, že \(\frac{\mathrm{d}\vect{v}}{|\mathrm{d}\vect{v}|} = \vect{n}\) je normála plochy \(S\):

\begin{equation}
P = -\lambda \cdot S \cdot \vect{n} \cdot \grad \ \mathrm{t}
\end{equation}

Tepelný tok nekonečně malou plochou \(\mathrm{d}S\) je pak:

\begin{equation}
\mathrm{d}P = -\lambda \cdot \mathrm{d}S \cdot \vect{n} \cdot \grad \ \mathrm{t} = -\lambda \cdot \mathrm{d}\vect{s} \cdot \grad \ \mathrm{t}
\end{equation}

A~celkový tok obecnou plochou \(S\) je pak:

\begin{equation}
P = -\lambda \cdot \int_{S} \grad \ \mathrm{t} \cdot \mathrm{d}\vect{s} 
\end{equation}

Je-li plocha \(S\) uzavřená obklopující těleso \(V\), pak tento výkon musí být roven celkovému výkonu zdroje v~této ploše. Označme hustotu výkonu zdroje \(p_z\). Pak:

\begin{equation}
\int_{V} p_z \mathrm{d}V = -\lambda \cdot \oint_{\partial V} \grad \ \mathrm{t} \cdot \mathrm{d}\vect{s} 
\end{equation} 

Na plošný integrál na pravé straně rovnice použijeme Gaussův teorém: 

\begin{equation}
\int_{V} p_z \mathrm{d}V = -\lambda \cdot \int_{V} \diverg \ \grad \ \mathrm{t} \ \mathrm{d}V = \int_{V} -\lambda \cdot \diverg \ \grad \ \mathrm{t} \ \mathrm{d}V
\end{equation}

Má-li tato rovnice platit pro libovolný objem \(V\), pak musí platit v~každém bodě. Můžeme tedy odstranit objemové integrály:

\begin{equation}
p_z = -\lambda \cdot \diverg \ \grad \ t = -\lambda \cdot \Delta \mathrm{t}
\end{equation}

Tím jsme získaly Poissonovu rovnici pro vedení tepla.

\begin{fact}

\begin{equation}
p_z = -\lambda \cdot \Delta \mathrm{t}
\end{equation}

\begin{equation}
P = -\lambda \cdot \int_{S} \grad \ \mathrm{t} \cdot \mathrm{d}\vect{s} 
\end{equation}

\(P\) - tepelný tok [\(\mathrm{W}\)]

\(p_z\) - hustota tepelného toku z~vnějšího zdroje [\(\mathrm{W} \mathrm{m}^{-3}\)]

\(t\) - teplota [\(\mathrm{K}, \si{\degree}\mathrm{C}\)]

\(\lambda\) - tepelná vodivost [\(\mathrm{W} \mathrm{m}^{-1} \mathrm{K}^{-1}\)] 
\end{fact}

\section{Proud ve vodiči}

Začněme ohmovým zákonem:

\begin{equation}
\label{eq:proud_ve_vodici_1}
I = \frac{U}{R}
\end{equation}

Odpor vodiče o~délce \(l\), průřezu \(S\) a~měrné vodivosti \(sigma\) je určen vztahem \(R = \frac{l}{\sigma S}\). Ten můžeme dosadit do předešlé rovnice. Dále zaveďme elektrický potenciál \(\varphi\). Eletrické napětí \(U\) mezi dvěma body je rovno rozdílu potenciálů v~těchto bodech, tedy \(U = -\mathrm{d}\varphi\). Záporné znaménko značí, že proud teče z~místa s~větším potenciálem do místa s~nižším potenciálem:

\begin{equation}
\label{eq:proud_ve_vodici_2}
I = -\sigma S \frac{-\mathrm{d}\varphi}{l}
\end{equation}

Srovnejme rovnici \eqref{eq:proud_ve_vodici_2} s~rovnicí \eqref{eq:tepelny_tok_1}. Vidíme, že se liší pouze ve veličinách, jinak mají stejnou formu. Proto i~postup jejich řešení bude shodný. Postup proto nebudeme opakovat.

\begin{fact}

\begin{equation}
j_z = -\sigma \cdot \Delta \mathrm{\varphi}
\end{equation}

\begin{equation}
I = -\sigma \cdot \int_{S} \grad \ \mathrm{\varphi} \cdot \mathrm{d}\vect{s} 
\end{equation}

\begin{equation}
U_{AB} = \varphi(A) - \varphi(B) 
\end{equation}

\(I\) - proud [\(\mathrm{A}\)]

\(p_z\) - proudová hustota vnějšího zdroje proudu [\(\mathrm{A} \mathrm{m}^{-3}\)]

\(\varphi\) - potenciál elektrického pole [\(\mathrm{V}\)]

\(U\) - elektrické napětí [\(\mathrm{V}\)]

\(\sigma\) - elektrická vodivost [\(\mathrm{A} \mathrm{m}^{-1} \mathrm{V}^{-1}\)] 
\end{fact}


\section{Centrální síla klesající s~kvadrátem vzdálenosti}

\subsection{Elektrostatické pole}
\subsection{Gravitační pole}

\section{Stacionární magnetické pole}

\section{Elektromagnetické pole}
