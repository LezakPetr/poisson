\chapter{Fyzikální jevy}

Poissonova rovnice umožňuje řešit celou řadu různých fyzikálních jevů. Níže si některé z~nich popíšeme. V~některých případech
je ale nutné fyzikální rovnici napřed upravit na Poissonovu rovnici.

\section{Tepelný tok}

Mějme izotropní tepelně vodivý materiál. 
Máme-li v materiálu 2 plochy o~obsahu \(S\) ve vzdálenosti \(v\) s teplotním rozdílem \(\mathrm{d}t\), pak mezi nimi bude proudit
tepelný tok \(P = -\lambda S \frac{\mathrm{d}t}{v}\). Záporné znaménko značí, že směr proudu tepla je opačný než směr růstu
teploty - teplo proudí z~místa s~vyšší teplotou do místa s~nižší teplotou.


\section{Elektrostatické pole}

\section{Stacionární magnetické pole}

\section{Elektromagnetické pole}
