\chapter{Fyzikální jevy}

Poissonova rovnice umožňuje řešit celou řadu různých fyzikálních jevů. Níže si některé z~nich popíšeme. V~některých případech
je ale nutné fyzikální rovnici napřed upravit na Poissonovu rovnici.

\section{Tepelný tok}

Mějme izotropní tepelně vodivý materiál s~tepelnou vodivostí \(\lambda\). 
Máme-li v materiálu 2 plochy o~obsahu \(S\) ve vzdálenosti \(v\) s~teplotním rozdílem \(\mathrm{d}t\), pak mezi nimi bude proudit
tepelný tok \(P = -\lambda S \frac{\mathrm{d}t}{v}\). Záporné znaménko značí, že směr proudu tepla je opačný než směr růstu
teploty - teplo proudí z~místa s~vyšší teplotou do místa s~nižší teplotou. 

Zapišme tento vztah vektorově. 
Začneme tím, že nahradíme vzdálenost \(v\) vektorem nekonečně malého posunutí \(\mathrm{d}\vect{v}\). Dále podle vztahu \eqref{eq:zmena_pole_gradient} platí \(\mathrm{d}t = \mathrm{t}(x + \mathrm{d}v_x, y + \mathrm{d}v_y, z + \mathrm{d}v_z) - t(x, y, z) = \mathrm{d}\vect{v} \cdot \grad \ \mathrm{t}\):

\begin{equation}
P = -\lambda S \frac{\mathrm{d}\vect{v} \cdot \grad \ \mathrm{t}}{|\mathrm{d}\vect{v}|}
\end{equation}

Vidíme, že \(\frac{\mathrm{d}\vect{v}}{|\mathrm{d}\vect{v}|} = \vect{n}\) je normála plochy \(S\):

\begin{equation}
P = -\lambda \cdot S \cdot \vect{n} \cdot \grad \ \mathrm{t}
\end{equation}

Tepelný tok nekonečně malou plochou \(\mathrm{d}S\) je pak:

\begin{equation}
\mathrm{d}P = -\lambda \cdot \mathrm{d}S \cdot \vect{n} \cdot \grad \ \mathrm{t} = -\lambda \cdot \mathrm{d}\vect{s} \cdot \grad \ \mathrm{t}
\end{equation}

A~celkový tok obecnou plochou \(S\) je pak:

\begin{equation}
P = -\lambda \cdot \int_{S} \grad \ \mathrm{t} \cdot \mathrm{d}\vect{s} 
\end{equation}

Je-li plocha \(S\) uzavřená obklopující těleso \(V\), pak tento výkon musí být roven celkovému výkonu zdroje v~této ploše. Označme hustotu výkonu zdroje \(p_z\). Pak:

\begin{equation}
\int_{V} p_z \mathrm{d}V = -\lambda \cdot \oint_{\partial V} \grad \ \mathrm{t} \cdot \mathrm{d}\vect{s} 
\end{equation} 

Na plošný integrál na pravé straně rovnice použijeme Gaussův teorém: 

\begin{equation}
\int_{V} p_z \mathrm{d}V = -\lambda \cdot \int_{V} \diverg \ \grad \ \mathrm{t} \ \mathrm{d}V = \int_{V} -\lambda \cdot \diverg \ \grad \ \mathrm{t} \ \mathrm{d}V
\end{equation}

Má-li tato rovnice platit pro libovolný objem \(V\), pak musí platit v~každém bodě. Můžeme tedy odstranit objemové integrály:

\begin{equation}
p_z = -\lambda \cdot \diverg \ \grad \ t = -\lambda \cdot \Delta \mathrm{t}
\end{equation}

Tím jsme získaly Poissonovu rovnici pro vedení tepla.

\section{Proud ve vodiči}

\section{Centrální síla klesající s~kvadrátem vzdálenosti}

\subsection{Elektrostatické pole}
\subsection{Gravitační pole}

\section{Stacionární magnetické pole}

\section{Elektromagnetické pole}
