\chapter{Fyzikální jevy}

Poissonova rovnice umožňuje řešit celou řadu různých fyzikálních jevů. Níže si některé z~nich popíšeme. V~některých případech
je ale nutné fyzikální rovnici napřed upravit na Poissonovu rovnici.

\section{Tepelný tok}

Mějme izotropní tepelně vodivý materiál s~tepelnou vodivostí \(\lambda\). 
Máme-li v materiálu 2 plochy o~obsahu \(S\) ve vzdálenosti \(v\) s~teplotním rozdílem \(\mathrm{d}t\), pak mezi nimi bude proudit tepelný tok 

\begin{equation}
\label{eq:tepelny_tok_1}
P = -\lambda S \frac{\mathrm{d}t}{v}
\end{equation}

Záporné znaménko značí, že směr proudu tepla je opačný než směr růstu
teploty - teplo proudí z~místa s~vyšší teplotou do místa s~nižší teplotou. 

Zapišme tento vztah vektorově. 
Začneme tím, že nahradíme vzdálenost \(v\) vektorem nekonečně malého posunutí \(\mathrm{d}\vect{v}\). Dále podle vztahu \eqref{eq:zmena_pole_gradient} platí \(\mathrm{d}t = \mathrm{t}(x + \mathrm{d}v_x, y + \mathrm{d}v_y, z + \mathrm{d}v_z) - t(x, y, z) = \mathrm{d}\vect{v} \cdot \grad \ \mathrm{t}\):

\begin{equation}
P = -\lambda S \frac{\mathrm{d}\vect{v} \cdot \grad \ \mathrm{t}}{|\mathrm{d}\vect{v}|}
\end{equation}

Vidíme, že \(\frac{\mathrm{d}\vect{v}}{|\mathrm{d}\vect{v}|} = \vect{n}\) je normála plochy \(S\):

\begin{equation}
P = -\lambda \cdot S \cdot \vect{n} \cdot \grad \ \mathrm{t}
\end{equation}

Tepelný tok nekonečně malou plochou \(\mathrm{d}S\) je pak:

\begin{equation}
\mathrm{d}P = -\lambda \cdot \mathrm{d}S \cdot \vect{n} \cdot \grad \ \mathrm{t} = -\lambda \cdot \mathrm{d}\vect{s} \cdot \grad \ \mathrm{t}
\end{equation}

A~celkový tok obecnou plochou \(S\) je pak:

\begin{equation}
P = -\lambda \cdot \int_{S} \grad \ \mathrm{t} \cdot \mathrm{d}\vect{s} 
\end{equation}

Je-li plocha \(S\) uzavřená obklopující těleso \(V\), pak tento výkon musí být roven celkovému výkonu zdroje v~této ploše. Označme hustotu výkonu zdroje \(p_z\). Pak:

\begin{equation}
\int_{V} p_z \mathrm{d}V = -\lambda \cdot \oint_{\partial V} \grad \ \mathrm{t} \cdot \mathrm{d}\vect{s} 
\end{equation} 

Na plošný integrál na pravé straně rovnice použijeme Gaussův teorém: 

\begin{equation}
\int_{V} p_z \mathrm{d}V = -\lambda \cdot \int_{V} \diverg \ \grad \ \mathrm{t} \ \mathrm{d}V = \int_{V} -\lambda \cdot \diverg \ \grad \ \mathrm{t} \ \mathrm{d}V
\end{equation}

Má-li tato rovnice platit pro libovolný objem \(V\), pak musí platit v~každém bodě. Můžeme tedy odstranit objemové integrály:

\begin{equation}
p_z = -\lambda \cdot \diverg \ \grad \ t = -\lambda \cdot \Delta \mathrm{t}
\end{equation}

Tím jsme získaly Poissonovu rovnici pro vedení tepla.

\begin{fact}

\begin{equation}
p_z = -\lambda \cdot \Delta \mathrm{t}
\end{equation}

\begin{equation}
P = -\lambda \cdot \int_{S} \grad \ \mathrm{t} \cdot \mathrm{d}\vect{s} 
\end{equation}

\(P\) - tepelný tok [\(\mathrm{W}\)]

\(p_z\) - hustota tepelného toku z~vnějšího zdroje [\(\mathrm{W} \cdot \mathrm{m}^{-3}\)]

\(t\) - teplota [\(\mathrm{K}, \si{\degree}\mathrm{C}\)]

\(\lambda\) - tepelná vodivost [\(\mathrm{W} \cdot \mathrm{m}^{-1} \cdot \mathrm{K}^{-1}\)] 
\end{fact}

\section{Proud ve vodiči}

Začněme ohmovým zákonem:

\begin{equation}
\label{eq:proud_ve_vodici_1}
I = \frac{U}{R}
\end{equation}

Odpor vodiče o~délce \(l\), průřezu \(S\) a~měrné vodivosti \(sigma\) je určen vztahem \(R = \frac{l}{\sigma S}\). Ten můžeme dosadit do předešlé rovnice. Dále zaveďme elektrický potenciál \(\varphi\). Eletrické napětí \(U\) mezi dvěma body je rovno rozdílu potenciálů v~těchto bodech, tedy \(U = -\mathrm{d}\varphi\). Záporné znaménko značí, že proud teče z~místa s~větším potenciálem do místa s~nižším potenciálem:

\begin{equation}
\label{eq:proud_ve_vodici_2}
I = -\sigma S \frac{-\mathrm{d}\varphi}{l}
\end{equation}

Srovnejme rovnici \eqref{eq:proud_ve_vodici_2} s~rovnicí \eqref{eq:tepelny_tok_1}. Vidíme, že se liší pouze ve veličinách, jinak mají stejnou formu. Proto i~postup jejich řešení bude shodný. Postup proto nebudeme opakovat.

\begin{fact}
\begin{equation}
j_z = -\sigma \cdot \Delta \mathrm{\varphi}
\end{equation}

\begin{equation}
I = -\sigma \cdot \int_{S} \grad \ \mathrm{\varphi} \cdot \mathrm{d}\vect{s} 
\end{equation}

\begin{equation}
U_{AB} = \varphi(A) - \varphi(B) 
\end{equation}

\(I\) - proud [\(\mathrm{A}\)]

\(j_z\) - proudová hustota vnějšího zdroje proudu [\(\mathrm{A} \cdot \mathrm{m}^{-3}\)]

\(\varphi\) - potenciál elektrického pole [\(\mathrm{V}\)]

\(U\) - elektrické napětí [\(\mathrm{V}\)]

\(\sigma\) - elektrická vodivost [\(\mathrm{A} \cdot \mathrm{m}^{-1} \cdot \mathrm{V}^{-1}\)] 
\end{fact}


\section{Centrální síla klesající s~kvadrátem vzdálenosti}

\subsection{Elektrostatické pole}
\subsection{Gravitační pole}

\section{Stacionární magnetické pole}

Začněme Ampérovým zákonem:

\begin{equation}
\oint_{\partial S} \vect{B} \cdot \mathrm{d}\vect{l} = \mu \cdot I
\end{equation}

Proud \(I\) představuje celkový proud tekoucí uvnitř (libovolné) plochy \(S\), která má hranici \(\partial S\). Neboli, cirkulace vektoru magnetické indukce \(B\) okolo libovolné uzavřené křivky je rovna celkovému proudu, který tato křivka obepíná, násobenému permeabilitou prostředí \(\mu\). Zavedeme proto plošnou hustotu proudu \(j\) a~proud vyjádříme pomocí ní:

\begin{equation}
\oint_{\partial S} \vect{B} \cdot \mathrm{d}\vect{l} = \int_S \mu \cdot \vect{j} \cdot \mathrm{d}\vect{s}
\end{equation}

Na integrál na levé straně rovnice použijeme Stokessovu větu:

\begin{equation}
\int_{S} \rot \ \vect{B} \cdot \mathrm{d}\vect{s} = \int_S \mu \cdot \vect{j} \cdot \mathrm{d}\vect{s}
\end{equation}

Protože plocha \(S\) může být libovolná, tak musí být rovny i~integrované funkce:

\begin{equation}
\rot \ \vect{B} = \mu \cdot \vect{j}
\end{equation}

Touto rovnicí ale není magnetická indukce \(B\) určena jednoznačně. Musíme připojit rovnici udávající, že megnetická indukce je vírové pole. Neexisují tedy zřídla magnetického pole, magnetické monopóly:

\begin{equation}
\diverg \ \vect{B} = 0
\end{equation}

My bychom potřebovali obě rovnice spojit do jedné. Toho můžeme dosáhnout tak, že zavedeme vektorový potenciál \(A\):

\begin{equation}
\vect{B} = \rot \ \vect{A}
\end{equation}

Tím máme zajištěno, že platí

\begin{equation}
\diverg \ \vect{B} = \diverg \ \rot \ \vect{A} = 0
\end{equation}

a~máme tak jedinou rovnici:

\begin{equation}
\rot \ \rot \ \vect{A} = \mu \cdot \vect{j}
\end{equation}

Vektorový potenciál ale není určen jednoznačně. Může proto zavést podmínku 

\begin{equation}
\diverg \ \vect{A} = 0
\end{equation}

Blíže je toto zavedení podmínky popsáno v~sekci~\ref{sec:pole_virova}. Napišme znovu vztah~\eqref{eq:vect_laplace_rozepsany}:

\begin{equation}
\Delta \ \vect{A} = \grad \ \diverg \ \vect{A} - \rot \ \rot \vect{A} 
\end{equation}

Dosazením do pravé strany rovnice získáme vektorovou rovnici, která je v~kartézském souřadném systému řešitelná jako tři nezávislé Poissonovy rovnice pro jednotlivé složky:

\begin{equation}
\Delta \ \vect{A} = -\mu \cdot \vect{j}
\end{equation}

Při studiu elektromagnetické indukce je důležitá veličina magnetického toku plochou. Zaveďme ji tedy.

\begin{equation}
\Phi = \int_S \ \vect{B} \cdot \mathrm{d}\vect{s}
\end{equation}

Prozkoumejme souvislost toku \(\Phi\) s~vektorovým potenciálem \(A\). V~rovnici pomocí něj vyjádříme magnetickou indukci a~použijeme Stokessovu větu:

\begin{equation}
\Phi = \int_S \ \rot \ \vect{A} \cdot \mathrm{d}\vect{s} = \oint_{\partial S} \vect{A} \cdot \mathrm{d}\vect{l}
\end{equation}

\begin{fact}
\begin{equation}
\Delta \ \vect{A} = -\mu \cdot \vect{j}
\end{equation}

\begin{equation}
\vect{B} = \rot \ \vect{A}
\end{equation}

\begin{equation}
\Phi = \oint_{\partial S} \vect{A} \cdot \mathrm{d}\vect{l}
\end{equation}

\(j\) - proudová hustota [\(\mathrm{A} \cdot \mathrm{m}^{-3}\)]

\(\mu\) - permeabilita prostředí [\(\mathrm{H} \cdot \mathrm{m}^{-1}\)]

\(\vect{A}\) - vektorový potenciál magnetického pole [\(\mathrm{T} \cdot \mathrm{m}\)]

\(\vect{B}\) - magnetická indukce [\(\mathrm{T}\)]

\(\Phi\) - magnetický tok [\(\mathrm{Wb}\)]
\end{fact}


\section{Elektromagnetické pole}
