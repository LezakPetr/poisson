\chapter{Řešení Poissonovy rovnice na přímce}

Nejdříve si ukážeme, jak lze řešit Poissonovu rovnici na přímce. Tu dokážeme řešit analyticky, a ukážeme si, jak bude řešení probíhat.
Začněme tím, jak bude vypadat řešení rovnice.

Poissonovu rovnici můžeme zapsat ve tvaru \eqref{eq:reseni_primka_1a} a \eqref{eq:reseni_primka_1b}. V podstatě jsme jen zavedli označení pro první derivaci potenciálu.

\begin{equation}
\label{eq:reseni_primka_1a}
\frac{\mathrm{d} \Phi}{\mathrm{d} x} = f(x) \\
\end{equation}

\begin{equation}
\label{eq:reseni_primka_1b}
\frac{\mathrm{d} \varphi}{\mathrm{d} x} = \Phi(x)
\end{equation}


Pokud budeme integrovat rovnici \eqref{eq:reseni_primka_1a} od \(0\) do \(t\), získáme rovnici \eqref{eq:reseni_primka_2}, přesunutím členu pak rovnici \eqref{eq:reseni_primka_3}.

\begin{equation}
\label{eq:reseni_primka_2}
\Phi(t) - \Phi(0) = \int_0^t f(x) \mathrm{d}x
\end{equation}

\begin{equation}
\label{eq:reseni_primka_3}
\Phi(t) = \Phi(0) + \int_0^t f(x) \mathrm{d}x
\end{equation}

Integrováním rovnice \eqref{eq:reseni_primka_1b} od \(0\) do \(s\) \eqref{eq:reseni_primka_3} získáme rovnici \eqref{eq:reseni_primka_4}.

\begin{equation}
\label{eq:reseni_primka_4}
\varphi(x) - \varphi(0) = \int_0^x \Phi(s) \mathrm{d}s
\end{equation}

Nyní do rovnice \eqref{eq:reseni_primka_4} dosadíme rovnici \eqref{eq:reseni_primka_3} a získáme rovnici \eqref{eq:reseni_primka_5}. Z ní pak vyjádříme
\(\varphi(x)\) a získáme rovnici \eqref{eq:reseni_primka_6}.

\begin{equation}
\label{eq:reseni_primka_5}
\varphi(x) - \varphi(0) = \int_0^x \left( \Phi(0) + \int_0^s f(t) \mathrm{d}t \right) \mathrm{d}s = x \cdot \Phi(0) + \int_0^x \int_0^s f(t) \ \mathrm{d}t \ \mathrm{d}s
\end{equation}

\begin{equation}
\label{eq:reseni_primka_6}
\varphi(x) = \varphi(0) + x \cdot \Phi(0) + \int_0^x \int_0^s f(t) \ \mathrm{d}t \ \mathrm{d}s
\end{equation}

V této rovnici vystupují 2 parametry - potenciál na počátku \(\varphi(0)\) a derivace potenciálu na počátku \(\Phi(0)\). Rovnice \eqref{eq:reseni_primka_6} tak umožňuje
řešit Poissonovu rovnici s podmínkami, že potenciál na počátku nabývá hodnoty \(\varphi(0)\) a jeho první derivace hodnoty \(\Phi(0)\). Tyto podmínky nazýváme Cauchyovskými počátečními podmínkami.

Uvažujme nyní řešení na intervalu (0, \(L\)). Do rovnice \eqref{eq:reseni_primka_6} můžeme dosadit \(t = L\) a vyjádřit \(\Phi(0)\).
Získáme tak rovnici \eqref{eq:reseni_primka_7}.

\begin{equation}
\label{eq:reseni_primka_7}
\Phi(0) = \frac{\varphi(L) - \varphi(0) - \int_0^L \int_0^s f(t) \ \mathrm{d}t \ \mathrm{d}s}{L}
\end{equation}

Dosadíme-li ji do rovnice \eqref{eq:reseni_primka_6} získáme rovnici \eqref{eq:reseni_primka_8}.

\begin{equation}
\label{eq:reseni_primka_8}
\varphi(x) = \varphi(0) + \frac{x}{L} \left( \varphi(L) - \varphi(0) - \int_0^L \int_0^s f(t) \ \mathrm{d}t \ \mathrm{d}s \right) + \int_0^x \int_0^s f(t) \ \mathrm{d}t \ \mathrm{d}s
\end{equation}

V rovnici \eqref{eq:reseni_primka_8} vystupují parametry \(\varphi(0)\) a \(\varphi(L)\), tedy potenciál na počátku a na konci intervalu. Rovnice \eqref{eq:reseni_primka_8} tak umožňuje
řešit Poissonovu rovnici s podmínkami, že potenciál na počátku intervalu nabývá hodnoty \(\varphi(0)\) a na konci intervalu hodnoty \(\varphi(L)\). Takovéto podmínky, které ohraničují sledovanou oblast, nazýváme okrajovými.
Okrajovou podmínku, která určuje hodnotu potenciálu, nazýváme Dirichletovu, rovnice \eqref{eq:reseni_primka_8} je proto řešení Poissonovy rovnice se dvěma Dirichletovými podmínkami.

Rovnice \eqref{eq:reseni_primka_8} nám umožňuje vypočítat řešení na intervalu \(x \in < 0; L >\). Proto všechny integrály v rovnici \eqref{eq:reseni_primka_8} budou probíhat uvnitř tohoto intervalu, není tak potřeba
znát \(f\) mimo interval. To platí obecně - pro řešení parciálních diferenciálních rovnic v utčité oblasti musíme znát zdroje v této oblasti a okrajové podmínky po kraji uvažované oblasti. Případné zdroje
mimo oblast jsou už "započítány" v okrajových podmínkách.

Dále můžeme upravit rovnici \eqref{eq:reseni_primka_2} tak, že dosadíme \(t = L\) a vyjádříme \(\Phi(0)\). Získáme tak rovnici \eqref{eq:reseni_primka_9}.

\begin{equation}
\label{eq:reseni_primka_9}
\Phi(0) = \Phi(L) - \int_0^L f(x) \mathrm{d}x
\end{equation}

Dosadíme-li ji do rovnice \eqref{eq:reseni_primka_6} získáme rovnici \eqref{eq:reseni_primka_10}.

\begin{equation}
\label{eq:reseni_primka_10}
\begin{split}
\varphi(x) = \varphi(0) + x \cdot \left( \Phi(L) - \int_0^L f(x) \mathrm{d}x \right) + \int_0^x \int_0^s f(t) \ \mathrm{d}t \ \mathrm{d}s = \\
\varphi(0) + x \cdot \Phi(L) - x \cdot \int_0^L f(x) \mathrm{d}x + \int_0^x \int_0^s f(t) \ \mathrm{d}t \ \mathrm{d}s
\end{split}
\end{equation}

V rovnici \eqref{eq:reseni_primka_10} vystupují parametry \(\varphi(0)\) a \(\Phi(L)\), tedy potenciál na počátku a derivace potenciálu na konci intervalu. Rovnice \eqref{eq:reseni_primka_10} tak umožňuje
řešit Poissonovu rovnici s podmínkami, že potenciál na počátku intervalu nabývá hodnoty \(\varphi(0)\) a jeho první derivace na konci intervalu hodnoty \(\Phi(L)\).
Okrajovou podmínku, která určuje první derivaci potenciálu, nazýváme Neumannovou.

Mohli bychom i řešit rovnici, kde na počátku intervalu bude zadaná Neumannova podmínka a na konci Dirichletova podmínka. Tuto možnost zde nebudeme rozebírat, stačí totiž otočit zadání a použít rovnici \eqref{eq:reseni_primka_10}.
Dále by nás mohlo napadnout řešit rovnici se dvěma Neumannovými podmínkami. Podle rovnice \eqref{eq:reseni_primka_9} však tyto podmínky nebudou nezávislé, ve skutečnosti budeme mít pouze jednu Neumannovu podmínku a ta nám neurčí
potenciál \(\varphi\) jednoznačně. Bude-li totiž \(\varphi\) řešením Poissonovy rovnice, pak i \(\varphi + C\) bude jejím řešením. Potřebujeme proto určit hodnotu potenciálu alespoň v jednom budě - potřebujeme alespoň
jednu Dirichletovu okrajovou podmínku.

\subsection{Příklad}
\label{sec:priklad_chladic_pasek}

Dva tranzistory jsou umístěny na hliníkový pásek o rozměrech 8cm x 1.5cm x 3mm. Konce pásku jsou připevněny
na chladič, který má pokojovou teplotu 20\si{\degree}C. Každý tranzistor produkuje ztrátový výkon \(P_T = 10\mathrm{W}\) o kterém předpokládáme,
že je rovnoměrně rozložen na celé jejich šířce 1cm. Tranzistory jsou od sebe a od okraje pásku vzdáleny 2cm.
Tepelná vodivost hliníku je \(\lambda = 237 W m^{-1} K^{-1}\). Na jakou teplotu se tranzistory zahřejí?

Pásek má průřez \(S = 1.5 \cdot 10^{-2} \cdot 3 \cdot 10^{-3} = 4.5 \cdot 10^-5 \mathrm{m^2}\).

Máme-li na pásku dva průřezy ve vzdálenosti \(\mathrm{d}x\) s teplotním rozdílem \(\mathrm{d}T\), pak mezi nimi bude proudit
tepelný tok \(P = -\lambda S \frac{\mathrm{d}t}{\mathrm{d}x}\). Záporné znaménko značí, že směr proudu tepla je opačný než směr růstu
teploty - teplo proudí z místa s vyšší teplotou do místa s nižší teplotou.

Zároveň platí zákon zachování energie, v tomto případě zákon zachování tepla. Máme-li úsek pásku o délce \(L\), pak
v ustáleném stavu musí být celkový tepelný tok tímto úsekem nulový, jinak by se úsek zahříval nebo ochlazoval.
Tepelný tok do pásku vstupuje jednak přes jeho krajní průřezy od okolních částí pásku, jednak tepelný tok \(p_T(x)\) od tranzistorů.

Zapsáno matematicky \(P_{x+L} - P_{x} = \int_{x}^{x+L} p_T dt\). Derivováním tohoto vztahu podle \(x\) získáme vztah
\(\frac{\mathrm{d}P}{\mathrm{d}x} = p_T\) platný pro každý bod pásku.

Spojením výše uvedených vztahů získáme vztah \(p_T = -\lambda S \frac{\mathrm{d}^2 t}{\mathrm{d}x^2}\), neboli \(\frac{\mathrm{d}^2 t}{\mathrm{d}x^2} = -\frac{p_T}{\lambda S}\). Do je Poissonova rovnice s Dirichletovými okrajovými podmínkami. Obrázek \ref{img:pasek_vykon} ukazuje průběh tepelného toku
od tranzistorů. Řešení rovnice je na obrázku \ref{img:pasek_teplota}. Maximální teplota pásku je 43.4\si{\degree}C, tranzistory mohou mít teplotu vyšší
kvůli tepelnému odporu mezi nimi a páskem.

\begin{figure}
	\includegraphics[width=\linewidth]{examples/chladic_pasek_1.png}
	\caption{Ztrátový výkon}
	\label{img:pasek_vykon}
\end{figure}

\begin{figure}
	\includegraphics[width=\linewidth]{examples/chladic_pasek_2.png}
	\caption{Teplota pásku}
	\label{img:pasek_teplota}
\end{figure}

\subsection{Řešení integrací zdrojů}

Výše uvedený postup řešil Poissonovu rovnici na přímce jako obyčejnou diferenciální rovnici. Jeho výsledkem je jednoduché analytické řešení. Tento postup bohužel nelze zobecnit na více rozměrů. Proto si nyní ukážeme 
jiný způsob jak Poissonovu rovnici na přímce řešit. Tento postup půjde do určité míry zobecnit do více rozměrů.

Začněme určením potenciálu násobku Diracova pulsu. Mějme zdroj \(f\) definaný tak, že je nenulový pouze nekonečně blízko počátku \(x = 0\), a \(\int_{-\infty}^{+\infty} f(x) dx = F\). Integrováním rovnice
\eqref{eq:reseni_primka_1a} od \(-t\) do \(t\) získáme rovnici \eqref{eq:reseni_primka_bodove_1}. Využíváme přitom faktu, že pro \(t > 0\) interval \(<-t; t>\) obsahuje celý zdroj \(F\).

\begin{equation}
\label{eq:reseni_primka_bodove_1}
\Phi(t) - \Phi(-t) = \int_{-t}^{t} f(x) dx = F
\end{equation}

Pokud je zdroj symetrický, což můžeme u nekonečně tenkého pulsu předpokládat, pak musí být symetrický i potenciál. Proto můžeme předpokládat \(\Phi(-t) = -\Phi(t)\). Získáme tak rovnici
\eqref{eq:reseni_primka_bodove_2}. Jejím integrováním získáme rovici \eqref{eq:reseni_primka_bodove_3}.

\begin{equation}
\label{eq:reseni_primka_bodove_2}
\frac{\mathrm{d} \varphi}{\mathrm{d} x} = \Phi(t) = \frac{F}{2}
\end{equation}

\begin{equation}
\label{eq:reseni_primka_bodove_3}
\varphi(x) = \int \frac{F}{2} dx = \frac{F}{2} x + C
\end{equation}

Rovnice \eqref{eq:reseni_primka_bodove_3} platí pro \(x >= 0\). Potenciál musí být symetrický, proto integrační konstantu zvolíme \(C = 0\) a symetričnost zajistíme použitím absolutní hodnoty. Potenciál
\(F\)-násobku Diracova pulsu je proto určen rovnicí \eqref{eq:reseni_primka_bodove_4}.

\begin{equation}
\label{eq:reseni_primka_bodove_4}
\varphi(x) = \frac{F}{2} | x |
\end{equation}

Obecné rozložení můžeme chápat jako sérii posunutých násobků Diracových pulsů. Příspěvek od jendoho elementárního zdroje proto určuje rovnice \eqref{eq:reseni_primka_bodove_5}, její integrací získáme
rovnici \eqref{eq:reseni_primka_bodove_6}, která představuje řešení Poissonovy rovnice na přímce bez okrajových podmínek. Toto řešení není určeno jednoznačně - můžeme k němu přičíst libovolnou
funkci s nulovou druhou derivací a získáme jiné řešení rovnice.

\begin{equation}
\label{eq:reseni_primka_bodove_5}
d \varphi = \frac{f(t)}{2} | x - t| dt
\end{equation}

\begin{equation}
\label{eq:reseni_primka_bodove_6}
\varphi(x) = \frac{1}{2} \int_{-\infty}^{+\infty} f(t) | x - t| dt
\end{equation}

\section{Řešení Laplaceovy rovnice na úsečce}

Připomeňme si, že Laplaceova rovnice je Poissonova rovnice bez zdrojů. Na přímce je její rovnice \eqref{eq:laplace_primka_1}. Vidíme, že jejím řešením je jakákoli funkce s nulovou druhou derivací,
tedy lineární funkce. Laplaceovu rovnici proto nemá smysl řešit bez okrajových podmínek, jakákoli lineární funkce by jí vyhověla.
Obecně tedy bude řešení Laplaceovy rovnice vypadat podle vztahu \eqref{eq:laplace_primka_2}. Řešení (přímku) jsme zapsali jako lineární kombinaci
funkcí, které vyhovují Laplaceově rovnici. Zde máme takové funkce 2 - \(1 - \frac{x}{L}\) a \(\frac{x}{L}\). Funkce byli zvoleny tak, že nabývají
hodnoty 0 na jednom kraji intervalu a 1 na druhém. To nicméně není nutné, usnadní nám to ale řešení soustavy rovnic.

\begin{equation}
\label{eq:laplace_primka_1}
\frac{\mathrm{d}^2 \varphi}{\mathrm{d}x^2} = 0
\end{equation}

\begin{equation}
\label{eq:laplace_primka_2}
\varphi = a (1 - \frac{x}{L}) + b \frac{x}{L}
\end{equation}

Chceme-li řešit Laplaceovu rovnici na intervalu \(<0; L>\) s Dirichletovými nebo Neumannovými okrajovými podmínkami, musíme najít lineární funkci,
která bude okrajovým podmínkám vyhovovat. Abychom sjednotili Direchletovy a Neumannovy podmínky sjednotili, tak zavedeme lineární
okrajovou podmínku \eqref{eq:linearni_podminka}. Rozepíšeme-li ji na oba okraje úsečky, tak získáme soustavu rovnic \eqref{eq:linearni_podminka_2}.
V první rovnici jsme ještě otočili znaménko u Neumannovy podmínky, aby význam byla derivace směrem ven z intervalu.
Vyjádřením hodnot a derivací z rovnice \eqref{eq:laplace_primka_2} a dosazením získáme rovnici \eqref{eq:linearni_podminka_3}. Zavedli jsme při tom
označení \(n_0 = \frac{N_0}{L}\) a \(n_L = \frac{N_L}{L}\).
To je soustava 2 rovnic o 2 neznámých která má řešení \eqref{eq:linearni_podminka_4}.

\begin{equation}
\label{eq:linearni_podminka}
D \cdot \varphi + N \cdot \varphi' = H
\end{equation}

\begin{equation}
\label{eq:linearni_podminka_2}
\begin{split}
D_0 \cdot \varphi(0) - N_0 \cdot \varphi'(0) = H_0 \\
D_L \cdot \varphi(L) + N_L \cdot \varphi'(L) = H_L
\end{split}
\end{equation}

\begin{equation}
\label{eq:linearni_podminka_3}
\begin{split}
D_0 \cdot a + n_0 \cdot (b - a) = H_0 \\
D_L \cdot b + n_L \cdot (b - a) = H_L
\end{split}
\end{equation}

\begin{equation}
\label{eq:linearni_podminka_4}
\begin{split}
a = \frac{H_0 D_L + H_L n_0 + H_0 n_l}{D_0 D_L + n_0 D_L + D_0 n_L} \\
b = \frac{H_L D_0 + H_L n_0 + H_0 n_l}{D_0 D_L + n_0 D_L + D_0 n_L}
\end{split}
\end{equation}

\begin{fact}
Díky linearitě Laplaceovy rovnice můžeme řešení zapsat jako lineární kombinaci funkcí, které rovnici vyhovují, a hledat
koeficienty této lineární kombinace.
\end{fact}

\section{Řešení Poissonovy rovnice na úsečce}

Shrňme si fakta o Poissonově a Laplaceově rovnici na přímce. Metodou zdrojů umíme řešit Poissonovu rovnici bez okrajových podmínek. Její řešení
zůstává v platnosti pokud k němu přičteme lineární funkci. Řešením Laplaceovy rovnice je lineární funkce.

Z toho nám vyplývá následující postup řešení Poissonovy rovnice na úsečce:

\begin{itemize}
\item Vyřešíme Poissonovu rovnici na celé přímce bez okrajových podmínek. Můžeme (ale nemusíme) při tom ignorovat zdroje mimo interval \(<0; L>\).
Označme toto řešení \(\varphi_P\).

\item Upravíme okrajové podmínky tak, že zohledníme příspěvek od \(\varphi_P\).

\item Vyřešíme Laplaceovu rovnici pro upravené okrajové podmínky. Označme její řešení \(\varphi_L\).

\item Řešením Poissonovy rovnice na intervalu \(<0; L>\) je \(\varphi = \varphi_P + \varphi_L\).
\end{itemize}

\subsection{Příklad}

Vyřešme příklad \eqref{sec:priklad_chladic_pasek} metodou integrace zdrojů. Při jeho řešení jsme odvodili rovnici \(\frac{\mathrm{d}^2 t}{\mathrm{d}x^2} = -\frac{p_T}{\lambda S}\). Funkce \(f(x) = -\frac{p_T(x)}{\lambda S}\) je po částech lineární funkce.
Řešení Poissonovy rovnice je \(\varphi_P(x) = \frac{1}{2} \int_{-\infty}^{+\infty} f(s) | x - s| ds\). Integrál můžeme rozepsat na součet dvou integrálů
a zbavit se tak absolutní hodnoty. Získáme tak řešení \(\varphi_P(x) = \frac{1}{2} \left(\int_{-\infty}^{x} f(s) (x - s) ds - \int_{x}^{+\infty} f(s) (x - s) ds \right)\). Pokud je funkce \(f\) dostatečně jednoduchá, tak můžeme tento integrál vypočítat analyticky. 

Je-li funkce \(f\) po částech lineární funkce, tedy funkce, která v bodech \(s_i\) nabývá hodnoty \(y_i\) a mezi sousedními body je lineární, tak můžeme
integrál přepsat na součet integrálů lineárních úseků \(\int_{s_1}^{s_n} f(s) (x - s) ds = \sum_{i=1}^{n-1} \int_{s_i}^{s_i} f(s) (x - s) ds\).
Integrál přes jeden úsek můžeme vyjádřit \(\int_{s_i}^{s_i} f(s) (x - s) ds = (s_{i+1} - s_i) \cdot \left(\frac{1}{2} \cdot (x - s_i) \cdot (y_{i+1} + y_i) - (s_{i+1} - s_i) \cdot (\frac{y_{i+1}}{3} + \frac{y_{i}}{6} \right)\). Odvození nechám na čtenáři.

\begin{figure}
	\includegraphics[width=\linewidth]{examples/chladic_pasek_integrace_zdroju_1.png}
	\caption{Řešení bez okrajových podmínek}
	\label{img:pasek_integrace_zdroju_1}
\end{figure}

