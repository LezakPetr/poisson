\chapter{Logika}

Logika je nauka o~odvozování tvrzení z~jiných tvrzení. V~této kapitole budou představeny základy výrokové a~predikátové logiky prvního řádu, aby byly čtenářům zřejmé formule používané dále v~knize.

Výrok je jakékoli tvrzení, o~kterém má smysl řící, že je pravdivé nebo nepravdivé. Výroky jsou proto:

\begin{itemize}
	\item 2 je sudé číslo. (pravdivý výrok)
	\item \(4 > 5\) (nepravdivý výrok)
	\item Počet planet ve vesmíru je dělitelný třemi. (výrok, jehož pravdivost nedokážeme určit)
	\item Pro každé sudé číslo \(n\) platí, že \(n + 1\) je liché číslo. (výrok s~vázanou proměnnou)
\end{itemize}

Zavedeme konvenci, že pravdivý výrok budeme označovat \(\true\) a~nepravdivý výrok \(\false\).

Predikát, někdy nezývaný výroková funkce, je výraz obsahující volné proměnné, ze které se dosazením za tyto volné proměnné stane výrok. Příklady predikátů jsou:

\begin{itemize}
	\item \(n\) je sudé číslo.
	\item \(x > y\)
\end{itemize}

Vidíme, že o~těchto predikátech nemá smysl řící, zda jsou pravdivé nebo ne. Avšak například dosazením \(n = 2\) do prvního predikátu získáme pravdivý výrok "2 je sudé číslo", zatímco dosazením \(n = \mathrm{automobil}\) získáme nepravdivý výrok "automobil je sudé číslo". Z~důvodu obecnosti můžeme jakýkoli výrok považovat za predikát s~nula volnými proměnnými.

Volné proměnné je nutné rozlišovat od vázaných proměnných. Volné proměnné nejsou v~predikátech nijak kvatnifikovány a~vstupují do nich "zvenku" jako parametry. Například již zmíněný predikát bychom mohli zapsat takto:

\begin{equation}
\mathrm{A}(n) = n \ \text{je sudé číslo}
\end{equation}

Naproti tomu výraz "Pro každé sudé číslo \(n\) platí, že \(n + 1\) je liché číslo" je výrokem. Proměnná \(n\) je zde kvantifikovaná tvrzením "pro každé sudé číslo \(n\)" a~nevstupuje tedy jako parametr. Tento výrok můžeme vyhodnotit aniž bychom znali konkrétní hodnotu vázané proměnné \(n\).

\section{Negace}

Je-li \(\mathrm{A}\) predikát (nebo výrok), pak

\begin{equation}
\overline{\predicate{A}} 
\end{equation}

je predikát, který je pravdivý tehdy a jen tehdy, když \(A\) není pravdivý. Říkáme "není pravda \(\mathrm{A}\)".
Ekvivalenci můžeme vyjádřit pravdivostní tabulkou:

\eeq{truth_not}

Příklad: Není pravda, že je den. \(\overline{\mathrm{den}}\).

Je třeba upozornit, že negace není tzv. pravý opak, ale výrok, který je pravdivý právě tehdy, když není pravdivý výrok negovaný. Negací výroku "prší" proto není "svítí slunce", ale "neprší. Ne vždy, když neprší, tak svítí slunce - může být zataženo.

\section{Ekvivalence}

Jsou-li \(\mathrm{A}\) a~\(\mathrm{B}\) predikáty, pak

\begin{equation}
\predicate{A} \equivalent \predicate{B} 
\end{equation}

je pravdivý, pokud mají oba predikáty stejnou pravdivost - jsou oba pravdivé nebo oba nepravdivé. Říkáme "\(\mathrm{A}\) je pravdivý tehdy a~jen tehdy, pokud je pravdivý \(\mathrm{B}\)" nebo "\(\mathrm{A}\) je pravdivý právě tehdy, když je pravdivý \(\mathrm{B}\)".
Ekvivalenci můžeme vyjádřit pravdivostní tabulkou:

\eeq{truth_equiv}

Příklad: Den je tehdy a~jen tehdy, když není noc. \(\mathrm{Den} \equivalent \overline{\mathrm{noc}}\).

\section{Konjunkce}

Jsou-li \(\mathrm{A}\) a~\(\mathrm{B}\) predikáty, pak

\begin{equation}
\predicate{A} \land \predicate{B} 
\end{equation}

je pravdivý, pokud jsou oba výroky pravdivé. Říkáme "\(\mathrm{A}\) a~(zárověň) \(\mathrm{B}\)".
Konjunkci můžeme vyjádřit pravdivostní tabulkou:

\eeq{truth_and}

Příklad: Země je kulatá a~obíhá okolo Slunce. Země je kulatá \(\land\) Země obíhá okolo Slunce.


\section{Disjunkce}

Jsou-li \(\mathrm{A}\) a~\(\mathrm{B}\) predikáty, pak

\begin{equation}
\predicate{A} \lor \predicate{B}
\end{equation}

je pravdivý, pokud je pravdivý alespoň jeden z~nich (tedy i~oba). Říkáme "\(\mathrm{A}\) nebo \(\mathrm{B}\)".
Disjunkci můžeme vyjádřit pravdivostní tabulkou:

\eeq{truth_or}

Příklad: Buď je den nebo noc. Den \(\lor\) noc.

\section{Implikace}

Jsou-li \(\mathrm{A}\) a~\(\mathrm{B}\) predikáty, pak

\begin{equation}
\predicate{A} \impl \predicate{B}
\end{equation}

je pravdivý, pokud z~předpokladu \(A\) plyne závěr \(B\). Tedy pokud není splněn předpoklad \(A\) nebo je splněn předpoklad \(B\). Říkáme "z~\(\mathrm{A}\) plyne \(\mathrm{B}\)", "Pokud \(\mathrm{A}\) pak \(\mathrm{B}\)" nebo "\(\mathrm{A}\) implikuje \(\mathrm{B}\)".
Implikaci můžeme vyjádřit pravdivostní tabulkou:

\eeq{truth_impl}

Příklad: Pokud prší, pak je mokro. Prší \(\impl\) mokro.

Implikaci není možné obrátit. Z~výše uvedeného výroku tak nelze vyvodit "pokud je pokro, pak prší" - po cestě mohl například projet kropící vůz. Dále je poněkud neintuitivní, že je implikace pravdivá, pokud je nepravdivý předpoklad. Tedy pokud neprší, pak je výše uvedený výrok pravdivý ať už je mokro nebo ne. 



\eeq{de_morgan_or}