\chapter{Logika}

Logika je nauka o~odvozování tvrzení z~jiných tvrzení. V~této kapitole budou představeny základy výrokové a~predikátové logiky prvního řádu, aby byly čtenářům zřejmé formule používané dále v~knize.

Výrok je jakékoli tvrzení, o~kterém má smysl řící, že je pravdivé nebo nepravdivé. Výroky jsou proto:

\begin{itemize}
	\item 2 je sudé číslo. (pravdivý výrok)
	\item \(4 > 5\) (nepravdivý výrok)
	\item Počet planet ve vesmíru je dělitelný třemi. (výrok, jehož pravdivost nedokážeme určit)
	\item Pro každé sudé číslo \(n\) platí, že \(n + 1\) je liché číslo. (výrok s~vázanou proměnnou)
\end{itemize}

Zavedeme konvenci, že pravdivý výrok budeme označovat \(\true\) a~nepravdivý výrok \(\false\).

Predikát, někdy nezývaný výroková funkce, je výraz obsahující volné proměnné, ze které se dosazením za tyto volné proměnné stane výrok. Příklady predikátů jsou:

\begin{itemize}
	\item \(n\) je sudé číslo.
	\item \(x > y\)
\end{itemize}

Vidíme, že o~těchto predikátech nemá smysl řící, zda jsou pravdivé nebo ne. Avšak například dosazením \(n = 2\) do prvního predikátu získáme pravdivý výrok "2 je sudé číslo", zatímco dosazením \(n = \mathrm{automobil}\) získáme nepravdivý výrok "automobil je sudé číslo". Z~důvodu obecnosti můžeme jakýkoli výrok považovat za predikát s~nula volnými proměnnými.

Volné proměnné je nutné rozlišovat od vázaných proměnných. Volné proměnné nejsou v~predikátech nijak kvatnifikovány a~vstupují do nich "zvenku" jako parametry. Například již zmíněný predikát bychom mohli zapsat takto:

\begin{equation}
\predicate{A}(n) = n \ \text{je sudé číslo}
\end{equation}

Naproti tomu výraz "Pro každé sudé číslo \(n\) platí, že \(n + 1\) je liché číslo" je výrokem. Proměnná \(n\) je zde kvantifikovaná tvrzením "pro každé sudé číslo \(n\)" a~nevstupuje tedy jako parametr. Tento výrok můžeme vyhodnotit aniž bychom znali konkrétní hodnotu vázané proměnné \(n\).

\section{Negace}

Je-li \(\predicate{A}\) predikát (nebo výrok), pak

\begin{equation}
\overline{\predicate{A}} 
\end{equation}

je predikát, který je pravdivý tehdy a jen tehdy, když \(A\) není pravdivý. Říkáme "není pravda \(\predicate{A}\)".
Ekvivalenci můžeme vyjádřit pravdivostní tabulkou:

\etab{truth_not}

Příklad: Není pravda, že je den. \(\overline{\predicate{den}}\).

Je třeba upozornit, že negace není tzv. pravý opak, ale výrok, který je pravdivý právě tehdy, když není pravdivý výrok negovaný. Negací výroku "prší" proto není "svítí slunce", ale "neprší. Ne vždy, když neprší, tak svítí slunce - může být zataženo.

\section{Ekvivalence}

Jsou-li \(\predicate{A}\) a~\(\predicate{B}\) predikáty, pak

\begin{equation}
\predicate{A} \equivalent \predicate{B} 
\end{equation}

je pravdivý, pokud mají oba predikáty stejnou pravdivost - jsou oba pravdivé nebo oba nepravdivé. Říkáme "\(\predicate{A}\) je pravdivý tehdy a~jen tehdy, pokud je pravdivý \(\predicate{B}\)" nebo "\(\predicate{A}\) je pravdivý právě tehdy, když je pravdivý \(\predicate{B}\)".
Ekvivalenci můžeme vyjádřit pravdivostní tabulkou:

\etab{truth_equiv}

Příklad: Den je tehdy a~jen tehdy, když není noc. \(\predicate{Den} \equivalent \overline{\predicate{noc}}\).

\section{Konjunkce}

Jsou-li \(\predicate{A}\) a~\(\predicate{B}\) predikáty, pak

\begin{equation}
\predicate{A} \land \predicate{B} 
\end{equation}

je pravdivý, pokud jsou oba výroky pravdivé. Říkáme "\(\predicate{A}\) a~(zárověň) \(\predicate{B}\)".
Konjunkci můžeme vyjádřit pravdivostní tabulkou:

\etab{truth_and}

Příklad: Země je kulatá a~obíhá okolo Slunce. Země je kulatá \(\land\) Země obíhá okolo Slunce.


\section{Disjunkce}

Jsou-li \(\predicate{A}\) a~\(\predicate{B}\) predikáty, pak

\begin{equation}
\predicate{A} \lor \predicate{B}
\end{equation}

je pravdivý, pokud je pravdivý alespoň jeden z~nich (tedy i~oba). Říkáme "\(\mathrm{A}\) nebo \(\mathrm{B}\)".
Disjunkci můžeme vyjádřit pravdivostní tabulkou:

\etab{truth_or}

Příklad: Buď je den nebo noc. Den \(\lor\) noc.

\section{Implikace}

Jsou-li \(\predicate{A}\) a~\(\predicate{B}\) predikáty, pak

\begin{equation}
\predicate{A} \impl \predicate{B}
\end{equation}

je pravdivý, pokud z~předpokladu \(A\) plyne závěr \(B\). Tedy pokud není splněn předpoklad \(A\) nebo je splněn předpoklad \(B\). Říkáme "z~\(\predicate{A}\) plyne \(\predicate{B}\)", "Pokud \(\predicate{A}\) pak \(\predicate{B}\)" nebo "\(\predicate{A}\) implikuje \(\predicate{B}\)".
Implikaci můžeme vyjádřit pravdivostní tabulkou:

\etab{truth_impl}

Příklad: Pokud prší, pak je mokro. Prší \(\impl\) mokro.

Implikaci není možné obrátit. Z~výše uvedeného výroku tak nelze vyvodit "pokud je pokro, pak prší" - po cestě mohl například projet kropící vůz. Také není možné pouze negovat výroky v~impliaci. Z uvedeného výroku nelze tedy odvodit "pokud neprší, pak není mokro" - opět mohl projet kropící vůz. Ale je možné impliaci obrátit~a~negovat výroky, tedy odvodit "pokud není mokro, pak neprší", viz tautologie~\eqref{eq:impl_swap}. Dále je poněkud neintuitivní, že je implikace pravdivá, pokud je nepravdivý předpoklad. Tedy pokud neprší, pak je výše uvedený výrok pravdivý ať už je mokro nebo ne. 

\section{Množiny}

Množina je neuspořádaný soubor prvků. Zápisem

\begin{equation}
\{x: \predicate{A}(x)\}
\end{equation}

rozumíme množinu všech prvků \(x\), pro které je predikát \(\predicate{A}(x)\) pravdivý. Je-li \(M\) množina, pak můžeme zavést predikát

\begin{equation}
x \in M
\end{equation}

který je pravdivý tehdy a~jen tehdy, pokud je prvek \(x\) obsažen v~množině \(M\). Jeho negaci označme

\begin{equation}
x \notin M \equivalent \negation{x \in M}
\end{equation}

Jsou-li \(M_1\) a~\(M_2\) množiny, pak \(M_1 \cap M_2\) je jejich průnik a~obsahuje prvky, které jsou obsaženy v~obou množinách. Proto platí:

\eeq{intersection_definition}

Obdobně \(M_1 \cup M_2\) je sjednocení množin a~obsahuje prvky, které jsou obsaženy v~jedné nabo druhé množině. Proto platí:

\eeq{union_definition}

Rozdíl množin \(M_1 \setminus M_2\) obsahuje prvky, které jsou obsaženy v~množině \(M_1\) a~nejsou obsaženy v~množině \(M_2\). Proto platí:

\eeq{difference_definition}

\section{Obecný kvantifikátor}

Je-li \(\predicate{A}(x)\) predikát, pak

\begin{equation}
\forall x \ \predicate{A}(x)
\end{equation}

je výrok, který je pravdivý tehdy a~jen tehdy, když \(\predicate{A}\) je pravdivý pro všechny hodnoty \(x\). Říkáme "pro každé \(x\) platí \(\predicate{A}(x)\)". Obdobně, pokud \(\predicate{A}(x, y, ...)\) je predikát, pak

\begin{equation}
\forall x \ \predicate{A}(x, y, ...)
\end{equation}

je predikát, který je pravdivý tehdy a~jen tehdy, když \(\predicate{A}\) je pravdivý pro všechny hodnoty \(x\). Tentokrát však již není výrokem, protože obsahuje volné proměnné \(y, ...\). Proměnná \(x\) je ve výrazu vázaná, již nevstupuje jako parametr zvenčí.

V~uvedených příkladech může proměnná \(x\) nabývat jakékoli hodnoty - tedy například 1, -2.5, automobil atd. Často potřebujeme omezit proměnnou na určitou množinu hodnot. K~tomu použijeme zápis

\begin{equation}
\forall x \in \mathrm{S} \ \predicate{A}(x)
\end{equation}

který chápeme jako zkrácení zápisu

\begin{equation}
\forall x \ \left( x \in \mathrm{S} \impl \predicate{A}(x) \right)
\end{equation}

tedy "pro každé \(x\) z~množiny \(\mathrm{S}\) platí \(\predicate{A}(x)\)" je shodné s~"pro každé \(x\) platí, že pokud \(x\) je prvkem množiny \(\mathrm{S}\), pak platí \(\predicate{A}(x)\)".

Příklad: Každé přirozené číslo je větší než 0. 

\(\forall x \in \mathbb{N} \ x > 0\) neboli \(\forall x \ (x \in \mathbb{N} \impl x > 0)\).


\section{Existenční kvantifikátor}

Je-li \(\predicate{A}(x)\) predikát, pak

\begin{equation}
\exists x \ \predicate{A}(x)
\end{equation}

je výrok, který je pravdivý tehdy a~jen tehdy, když \(\predicate{A}\) je pravdivý alespoň pro jednu hodnotu \(x\). Říkáme "existuje \(x\) takové, že platí \(\predicate{A}(x)\)". Obdobně, pokud \(\predicate{A}(x, y, ...)\) je predikát, pak

\begin{equation}
\exists x \ \predicate{A}(x, y, ...)
\end{equation}

je predikát, který je pravdivý tehdy a~jen tehdy, když \(\predicate{A}\) je pravdivý alespoň pro jednu hodnotu \(x\). Tato hodnota ale může být závislá na volných proměnných \(y, ...\). Proměnná \(x\) je ve výrazu vázaná, již nevstupuje jako parametr zvenčí.

V~uvedených příkladech může proměnná \(x\) nabývat jakékoli hodnoty. Často potřebujeme omezit proměnnou na určitou množinu hodnot. K~tomu použijeme zápis

\begin{equation}
\exists x \in \mathrm{S} \ \predicate{A}(x)
\end{equation}

který chápeme jako zkrácení zápisu

\begin{equation}
\exists x \ \left( x \in \mathrm{S} \land \predicate{A}(x) \right)
\end{equation}

tedy "existuje \(x\) z~množiny \(\mathrm{S}\) takové, že platí \(\predicate{A}(x)\)" je shodné s~"existuje \(x\) takové, že \(x\) je prvkem množiny \(\mathrm{S}\) a~zároveň platí \(\predicate{A}(x)\)".

Příklad: Pro každé přirozené číslo existuje přirozené číslo větší. 

\(\forall x \in \mathbb{N} \ \exists y \ y > x\)

\section{Vyhodnocování logických výrazů}

Logické výrazy, někdy nazývané formule, jsou obdobné matematickým výrazům. Skládají se z~predikátů, proměnných a~logických operátorů oposaných výše. Podobně jako u~matematických výrazů se pořadí vyhodnocování řídí závorkami a~prioritou operátorů. Operátory se vyhodnocují v~pořadí:

\(\overline{A}\), \(\forall\), \(\exists\), \(\land\), \(\lor\), \(\impl\), \(\equivalent\)

Příklad - určete pravdivost predikátu pro \(x = 5, y = 10\):
\begin{equation}
x < 3 \lor y = 10 \land (x = 5 \lor x < 5)
\end{equation}

Nejdříve uzávorkujeme výraz podle priority operátorů a~pak jej postupně vyhodnotíme:

\begin{equation}
\begin{split}
x < 3 \lor (x \cdot y = 50 \land (x = 5 \lor x < 5)) = \\
5 < 3 \lor (5 \cdot 10 = 50 \land (5 = 5 \lor 5 < 5)) = \\
\false \lor (\true \land (\true \lor \false)) = \false \lor (\true \land \true) = \false \lor \true = \true 
\end{split}
\end{equation}

\section{Tautologie}

Tautologie jsou výroky, které jsou pravdivé bez ohladu na pravdivost výroků v~nich obsažených. Tautologie můžeme dokázat vyzkoušením všech možnosí, kterých mohou tyto výroky nabývat. Ideálně zkonstruováním pravdivostní tabulky.

Příklad: Nechť \(\predicate{A}\) a~\(\predicate{B}\) jsou výroky. Pak bez ohledu na jejich pravdivost platí:

\eeq{de_morgan_example}

Sestrojme pravdivostní tabulku, která vyhodnotí výrok~\eqref{eq:de_morgan_example} pro všechny možné kombinace výroků \(\predicate{A}\) a~\(\predicate{B}\). V~tabulce jsou uvedeny i~hodnoty pro podvýroky:

\etab{de_morgan_example}

Vidíme, že výrok~\eqref{eq:de_morgan_example} platí ve všech případech. Takto lze obecně dokazovat tautologie, které neobsahují kvantifikátory.

Máme-li tautologii ve formě ekvivalence a~máme-li výraz, který obsahuje jednu stranu ekvivalence, pak ji můžeme nahradit druhou stranou ekvivalence podobně, jako to děláme s~operátorem rovnosti při úpravách matematických výrazů. Například máme výrok

\begin{equation}
\overline{\predicate{C}} \lor \overline{\predicate{C} \lor (\predicate{D} \land \predicate{E})}
\end{equation}

ve kterém můžeme na část výrazu použít uvedenou tautologii se substitucí \(\predicate{A} = \predicate{C}\) a~\(\predicate{B} = \predicate{D} \land \predicate{E}\):

\begin{equation}
\overline{\predicate{C}} \lor \overline{\predicate{C}} \land \overline{\predicate{D} \land \predicate{E}}
\end{equation}

Využitím další tautologie~\eqref{eq:or_specific} se celý výraz zjednoduší na

\begin{equation}
\overline{\predicate{C}}
\end{equation}

a~proto můžeme napsat

\eeq{tautology_example}


\begin{fact}
\eeq{or_symmetry}
\eeq{and_symmetry}
\eeq{or_associativity}
\eeq{and_associativity}
\eeq{or_distributivity}
\eeq{and_distributivity}
\eeq{de_morgan_or}
\eeq{de_morgan_and}
\eeq{impl_definition}
\eeq{impl_usage}
\eeq{impl_transitivity}
\eeq{impl_swap}
\eeq{equiv_to_impl}
\eeq{double_negation}
\eeq{excluded_middle}
\eeq{and_both}
\eeq{or_specific}
\eeq{forall_not_eq_not_exists}
\eeq{exists_not_eq_not_forall}
\end{fact}

Tautologie~\eqref{eq:or_symmetry} a~\eqref{eq:and_symmetry} vyjadřují symetričnost disjunkce a~konjunkce, tedy možnost otočit pořadí operandů. Tautologie~\eqref{eq:or_associativity} a~\eqref{eq:and_associativity} pak jejich asociativitu, tedy možnost změnit pořadí vyhodnocování disjunkcí nebo konjunkcí. Tautologie~\eqref{eq:or_distributivity} a~\eqref{eq:and_distributivity} vyjadřují distributivitu disjunkce a~konjunkce, tedy možnost "roznásobit" výraz. Pomocí De-Morganových pravidel~\eqref{eq:de_morgan_or} a~\eqref{eq:de_morgan_and} lze posunou negaci skrz disjunkci a~konjunkci, při tom ale dojde k~záměně disjunkce za konjunkci a~naopak.

Dálen následují vztahy týkající se implikace. Tautologie~\eqref{eq:impl_definition} představuje možnost vyjádřit implikaci pomocí disjunkce. Tautologie~\eqref{eq:impl_usage} odpovídá logickému usuzování: pokud platí \(\predicate{A}\) a~z~\(\predicate{A}\) plyne \(\predicate{B}\), pak musí platit \(\predicate{B}\). Tautologie~\eqref{eq:impl_transitivity} vyjadřuje tranzitivitu implikace: pokud z~\(\predicate{A}\) plyne \(\predicate{B}\) a~z~\(\predicate{B}\) plyne \(\predicate{C}\), pak z~\(\predicate{A}\) plyne \(\predicate{C}\). Tautologie~\eqref{eq:impl_swap} ukazuje, že výroky v~implikaci nelze pouze negovat, ale je také nutné implikaci otočit. Tautologie~\eqref{eq:equiv_to_impl} vyjadřuje fakt, že ekvivalence znamená obousměrnou implikaci.

Tautologie~\eqref{eq:double_negation} se nazývá zákon dvojí negace a~vyjadřuje fakt, že dvojitou negací získáme původní výrok. Tautologie~\eqref{eq:excluded_middle} se nazývá zákon o~vyloučení třetího a~vyjadřuje fakt, že buď platí výrok, nebo jeho negace. Obdobně tautologie~\eqref{eq:and_both} vyjadřuje fakt, že nemůže platit zárověň výrok a~jeho negace.

Tautologie~\eqref{eq:or_specific} vyjadřuje fakt, že disjunkce výroku a~jeho více specifické verze je rovna tomuto výroku.

