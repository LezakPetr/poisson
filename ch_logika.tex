\chapter{Logika}

Logika je nauka o~odvozování tvrzení z~jiných tvrzení. V~této kapitole budou představeny základy výrokové a~predikátové logiky prvního řádu, aby byly čtenářům zřejmé formule používané dále v~knize.

Výrok je jakékoli tvrzení, o~kterém má smysl řící, že je pravdivé nebo nepravdivé. Výroky jsou proto:

\begin{itemize}
	\item 2 je sudé číslo. (pravdivý výrok)
	\item \(4 > 5\) (nepravdivý výrok)
	\item Počet planet ve vesmíru je dělitelný třemi. (výrok, jehož pravdivost nedokážeme určit)
	\item Pro každé sudé číslo \(n\) platí, že \(n + 1\) je liché číslo. (výrok s~vázanou proměnnou)
\end{itemize}

Zavedeme konvenci, že pravdivý výrok budeme označovat \(\true\) a~nepravdivý výrok \(\false\).

Predikát, někdy nezývaný výroková funkce, je výraz obsahující volné proměnné, ze které se dosazením za tyto volné proměnné stane výrok. Příklady predikátů jsou:

\begin{itemize}
	\item \(n\) je sudé číslo.
	\item \(x > y\)
\end{itemize}

Vidíme, že o~těchto predikátech nemá smysl řící, zda jsou pravdivé nebo ne. Avšak například dosazením \(n = 2\) do prvního predikátu získáme pravdivý výrok "2 je sudé číslo", zatímco dosazením \(n = \mathrm{automobil}\) získáme nepravdivý výrok "automobil je sudé číslo". Z~důvodu obecnosti můžeme jakýkoli výrok považovat za predikát s~nula volnými proměnnými.

Volné proměnné je nutné rozlišovat od vázaných proměnných. Volné proměnné nejsou v~predikátech nijak kvatnifikovány a~vstupují do nich "zvenku" jako parametry. Například již zmíněný predikát bychom mohli zapsat takto:

\begin{equation}
\mathrm{A}(n) = n \ \text{je sudé číslo}
\end{equation}

Naproti tomu výraz "Pro každé sudé číslo \(n\) platí, že \(n + 1\) je liché číslo" je výrokem. Proměnná \(n\) je zde kvantifikovaná tvrzením "pro každé sudé číslo \(n\)" a~nevstupuje tedy jako parametr. Tento výrok můžeme vyhodnotit aniž bychom znali konkrétní hodnotu vázané proměnné \(n\).

\section{Ekvivalence}

Jsou-li \(\mathrm{A}\) a~\(\mathrm{B}\) predikáty (nebo výroky), pak

\begin{equation}
\predicate{A} \equiv \predicate{B} 
\end{equation}

je mají oba predikáty stejnou pravdivost - jsou oba pravdivé nebo oba nepravdivé. Říkáme "\(\mathrm{A}\) je pravdivý tehdy a~jen tehdy, pokud je pravdivý \(\mathrm{B}\)" nebo "\(\mathrm{A}\) je pravdivý právě tehdy, když je pravdivý \(\mathrm{B}\)".
Ekvivalenci můžeme vyjádřit pravostní tabulkou:

\eeq{truth_equiv}

\section{Konjunkce}

\section{Disjunkce}

\section{Implikace}

\eeq{de_morgan_or}